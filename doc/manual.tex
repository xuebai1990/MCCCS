%%%%%%%%%%%%%%%%%%%%%%%%%%%%%%%%%%%%%%%%%%%%%%%%%%%%%%%%%%%%%%%%%%%%%%%%%%%%
% To create a PostScript or PDF file from this source file, first
% process it with LaTeX, then convert to ps or pdf:
%
% latex manual.tex
% dvips -P pdf -t letter -f manual.dvi -o manual.ps
% ps2pdf manual.ps manual.pdf
%
% To create html web pages, use latex2html
% latex2html -local_icons -split 5 -toc_depth 4 -toc_stars -t 'MCCCS 2.5 User Manual' -address http://maxwell8.chem.umn.edu -up_url http://maxwell8.chem.umn.edu -up_title 'Main MCCCS Page' manual
%%%%%%%%%%%%%%%%%%%%%%%%%%%%%%%%%%%%%%%%%%%%%%%%%%%%%%%%%%%%%%%%%%%%%%%%%%%%
%
% Document history:
%
% -original README file by NDZ 03/12/02
% -rewritten and converted to LaTeX by JMS 12/11/03
% -updated by JMS 2/26/04, 4/21/04
% -updated by NR 2/11/04
% -updated by KMB 01/10 -- version 2.2
% -updated by PB 04/12 -- version 2.3
% -updated by RKL 07/12 -- version 2.4
% -added FQ model instructions, BLE 05/15 -- version 2.5
%
%%%%%%%%%%%%%%%%%%%%%%%%%%%%%%%%%%%%%%%%%%%%%%%%%%%%%%%%%%%%%%%%%%%%%%%%%%%%
% LaTeX controls:
% {article}, {report}, {book}, {letter} = classes
% in [] formatting- font size, paper size... columns
\documentclass[12pt,letterpaper]{article}
%%%%%%%%%%%%%%%%%%%%%%%%%%%%%%%%%%%%%%%%%%%%%%%%%%%%%%%%%%%%%%%%%%%%%%%%%%%%

%%%%%%%%%%%%%%%%%%%%%%%%%%%%%%%%%%%%%%%%%%%%%%%%%%%%%%%%%%
% Import Packages
%%%%%%%%%%%%%%%%%%%%%%%%%%%%%%%%%%%%%%%%%%%%%%%%%%%%%%%%%%
% If figures are required
%\usepackage{epsf}
%\usepackage{epsfig}
\usepackage[hyphens]{url}
\usepackage{hyperref}
\usepackage{breakurl}
\usepackage{amsmath,amsfonts,amsthm,amssymb}
\usepackage{mathrsfs}
%%%%%%%%%%%%%%%%%%%%%%%%%%%%%%%%%%%%%%%%%%%%%%%%%%%%%%%%%%

%%%%%%%%%%%%%%%%%%%%%%%%%%%%%%%%%%%%%%%%%%%%%%%%%%%%%%%%%%
% Set Page Layout
%%%%%%%%%%%%%%%%%%%%%%%%%%%%%%%%%%%%%%%%%%%%%%%%%%%%%%%%%%
\hoffset=0in
\voffset=0in
\topmargin=-0.5in % topmargin is 1/2 inch (note negative value)
\headheight=0in
\headsep=0in
\footskip=15pt
\oddsidemargin=0in % left margin is 1 inch on right-hand pages
\evensidemargin=0in % same for left-hand pages in twosided document
\marginparsep=0in
\marginparwidth=0in
\textwidth=6.5in % leaves 1 inch for the right margin
\textheight=9.0in % 9 inches reserved for the text.
%%%%%%%%%%%%%%%%%%%%%%%%%%%%%%%%%%%%%%%%%%%%%%%%%%%%%%%%%%

\begin{document}

\parskip=6pt plus 1pt minus 1pt

%%%%%%%%%%%%%%%%%%%%%%%%%%%%%%%%%%%%%%%%%%%%%%%%%%%%%%%%%%%%%%%%%%%%%%%%%%%
%%%%%%%                       TITLE PAGE
%%%%%%%%%%%%%%%%%%%%%%%%%%%%%%%%%%%%%%%%%%%%%%%%%%%%%%%%%%%%%%%%%%%%%%%%%%%

\centerline{\bf \Huge MC$^3$S}

\vskip 24pt

\centerline{\bf \LARGE Monte Carlo for Complex Chemical Systems}

\vskip 24pt

\centerline{\bf \Large User Manual}

\vskip 24pt

\centerline{\bf \Large Version 2.5}

\vskip 36pt

\centerline{\large May 2015}

\eject

%%%%%%%%%%%%%%%%%%%%%%%%%%%%%%%%%%%%%%%%%%%%%%%%%%%%%%%%%%%%%%%%%%%%%%%%%%%
%%%%%%%                       BODY OF TEXT
%%%%%%%%%%%%%%%%%%%%%%%%%%%%%%%%%%%%%%%%%%%%%%%%%%%%%%%%%%%%%%%%%%%%%%%%%%%
%
% possible subdivisions
%
% \{}
% \subsection{}
% \subsubsection{}
% \paragraph{}
%
%%%%%%%%%%%%%%%%%%%%%%%%%%%%%%%%%%%%%%%%%%%%%%%%%%%%%%%
\section{Introduction}
This manual is intended to explain the variables necessary
for simulations carried out in the canonical ($NVT$),
isobaric-isothermal ($NpT$), $NVT$-Gibbs or $NpT$-Gibbs
ensemble. Simulations in the grand canonical ($\mu VT$)
ensemble are possible but not discussed.

%%%%%%%%%%%%%%%%%%%%%%%%%%%%%%%%%%%%%%%%%%%%%%%%%%%%%%%
\subsection{New Features in MCCCS Software Package}
\subsubsection{Version 2.5}
\begin{itemize}
\item Addition of variables for polarizable models
\end{itemize}

\subsubsection{Version 2.2}
\begin{itemize}
\item Addition of tabulated potentials
\item Addition of multiple-center rotations with {\bf iurot}
\item Addition of variables for RPLC simulations
\item Minor changes to the structure of the {\bf fort.4}
  input file
\item Introduction of the parallel version of the code
\item Introduction of the test suite
\end{itemize}

\subsubsection{Version 2.1}
\begin{itemize}
\item Automatic assignment of Ewald parameters and rcutchg,
  if requested
\item Thermodynamic integration
\item Expanded ensemble
\item Addition of external electric field
\item Calculation of solubility parameter, cohesive energy
  density, and heat of vaporization
\item Ability to use atom displacement
\item rigid swaps for fully flexible molecules
\item Torsion potential using tabulated torsional potential
\item Modifying the energy subroutines so that it doesn't
  use the neutral charge groups for Ewald summation
\item Ability to not use Coulombic part in boltzmann weights
\item Ability to use different pressures in different boxes,
  input pressure in MPa instead of simulation units
\item Ability to use ghost ideal particles in the
  simulation, a number of particle offset, to prevent
  unnecessary memory usage
\item Better way of excluding and including intra-molecular
  interactions
\item Bug fixes
\item System charge neutrality check at an appropriate place
\item Extensive checks for the correctness of the fortran.4
  input files for the molecule description part. Points the
  user where to look for a possible error while specifying
  LJ, stretch, bend, and torsion
\end{itemize}

%%%%%%%%%%%%%%%%%%%%%%%%%%%%%%%%%%%%%%%%%%%%%%%%%%%%%%%
\subsection{Code Development}
\subsubsection{Annotation}
Annotation of functionalities and algorithms should be
exercised, using the syntax of Doxygen
(\url{http://www.stack.nl/~dimitri/doxygen}) to allow for
the automatic generation of reference manual.

For each entity in the code there are two (or in some cases
three) types of descriptions, which together form the
documentation for that entity; the brief description and the
detailed description, both are optional. For subroutines and
functions there is also a third type of description, the
so-called in-body description, which consists of the
concatenation of all comment blocks found within the body of
the subroutine or function. As the name suggest, a brief
description is a short one-liner, whereas the detailed
description provides longer, more detailed documentation. An
"in-body" description can also act as a detailed description
or can describe a collection of implementation details. For
the HTML output brief descriptions are also used to provide
tooltips at places where an item is referenced.

\begin{itemize}
\item To mark a comment block as a detailed description,
  start it with ``!$>$'', and use ``!!'' or ``!$>$'' to
  continuate a one-line comment into a multi-line comment.
\item For the brief description, one could use the
  \textbackslash brief command with one of the above comment
  blocks. This command ends at the end of a paragraph, so
  the detailed description follows after an empty line.
\item If you want to document the members of a file or type,
  it is sometimes desirable to place the documentation block
  after the member instead of before it. For this purpose
  you can use ``!$<$'' to start the comment block. Note that
  this also works for the parameters of a function.
\item For functions one can use the \textbackslash param
  command to document their arguments and then use [in],
  [out], [in,out] to denote the direction. For inline
  documentation this is also possible by starting with the
  direction attribute.
\item Besides putting the comment blocks in front of the
  declaration or definition of a module, subroutine,
  function or type, or in front or after one of its members,
  you can put your documentation blocks elsewhere (except
  inside the body of a subroutine/function or a normal
  comment block) with the help of a structural command,
  which (like all other commands) start with a backslash
  (\textbackslash), followed by the command name and one or
  more parameters. Structural commands lead to some
  duplication of information, so you should avoid using them
  if possible. There are sometimes, however, reasons to use
  structural commands to place the documentation blocks in
  alternative places, for example, to document a file, when
  there is no such thing as "in front of a file".
\begin{description}
\item[\textbackslash file] to document a file.
\item[\textbackslash fn] to document a function.
\item[\textbackslash var] to document a variable.
\item[\textbackslash def] to document a \#define.
\item[\textbackslash typedef] to document a type definition.
\end{description}
\end{itemize}

Here is a example:
\begin{verbatim}
!> \brief Build the restriction matrix for the aggregation method.
!>
!> Detailed explanation of what the subroutine does.
!> Some more information,
!> And more...
!>
!> \param A,aggr information about the aggregates
!> (multiple arguments can be documented together using a comma
!> separated list)
!> \param A_ghost mask matrix for \b A (bold font A)
!> \return Restrict the restriction matrix
!> \remarks It uses advanced algorithms
!> \note This is only an example
!> \attention But you need to pay attention
!> \warning Otherwise it is bad
!> \bug None
!> \todo Handle special case
!> \author John Smith
!> \date 2012/04/01
function IntRestBuild(A,aggr,A_ghost) result(Restrict)
  implicit none
  Type(SpMtx),intent(in) ::A !< our fine level matrix
  Type(Aggrs),intent(in) ::aggr
  Type(SpMtx),intent(out)::A_ghost !< Our restriction matrix
\end{verbatim}

For more details on how to use Doxygen, please refer to its
documentation
(\url{http://www.stack.nl/~dimitri/doxygen/docblocks.html}).

\subsubsection{Test suite}
Test suite is available under the {\bf tests} sub-directory.
You should use it to check for errors after any non-trivial
modification to the code. Please see the {\bf README} file
for more information on the individual test cases.

\subsubsection{Management of code updates}
Management of code updates is done through the distributed
version control system, git (\url{http://git-scm.com}), and
there is a set of group guidelines
(\url{http://maxwell8.chem.umn.edu/index.php/Git_Guidelines})
which you should read and familiarize yourself with. Every
new feature as well as bug fix will need to be merged into
the group canonical code repository before your graduation,
so it is advisable to adopt its usage early in the process.

%%%%%%%%%%%%%%%%%%%%%%%%%%%%%%%%%%%%%%%%%%%%%%%%%%%%%%%
\subsection{Compiling and Running the Code}
\label{compile}
The many files that comprise the MCCCS software package are
organized into sub-directories, including {\bf doc} (tex
source file of the user manual), {\bf src}, (Fortran source
code), and {\bf tests}, and are available through git. It is
necessary to compile the code and set up all necessary input
files prior to running a simulation.

Before actually compiling the code, you first need to
generate a Makefile. CMake (\url{http://www.cmake.org}) is
currently used for this step. It is advisable to create a
separate directory, outside the root directory of the code
to avoid contaminating the latter and to better manage
executables on different machines that share the same file
system. You can select compilers by setting the FC
environment variable, and toggle the default precision of
floating number (DOUBLE\_PRECISION) and whether or not to
use MPI (USE\_MPI), OpenMP (USE\_OPENMP), and our own
implementation of certain math functions instead of using
library functions that are slower (USE\_OWN). You can also specify
the build type (CMAKE\_BUILD\_TYPE) to be either DEBUG, which will
write out additional information useful for determining where
errors are occurring, or RELEASE, which is the normal default mode. 
After running
cmake, execute command {\bf make} while still in the same
directory. To completely rebuild the code (only necessary
when switching to a different machine or after loading a
different compiler), execute {\bf make clean} followed by
{\bf make}. The code will be generated in {\bf src}
sub-directory. This often boils down to:
\begin{verbatim}
$ mkdir -p MCCCS/exe-itasca
$ cd MCCCS/exe-itasca
$ FC=ifort cmake -DDOUBLE_PRECISION=ON -DUSE_MPI=OFF \
-DUSE_OPENMP=OFF -DUSE_OWN=ON -DCMAKE_BUILD_TYPE=RELEASE ~/MCCCS/
# 1. assuming the code is in ~/MCCCS/ directory;
# 2. ccmake can be used in place of cmake for a graphical interface
$ make -j8 # build using 8 cores
\end{verbatim}

Vendor compilers often produce faster code; the {\bf ifort}
Intel compiler is often the choice for machines using Intel
processors, which include the latest Macs. Intel provides
the Linux version of its compiler family free of charge for
non-commercial uses
(\url{http://www.intel.com/software/products/noncom}). A set
of options have already been defined for a few common
compilers (GNU, G95, Intel, PGI, IBM XL, SGI MIPSpro, and
NAG) depending on whether or not it is a RELEASE or DEBUG build.

When preparing the input files, the sample files in the {\bf
  tests} sub-directory can be used as a starting point.
After setting all the variables in the input file(s) ({\bf
  topmon.inp}, {\bf fort.4}, and possibly others) and
verifying that all other required files are included, type
the name of the executable (be sure to include the path) to
run the simulation:
\begin{verbatim}
/path/to/topmon [--version|-v] [--help|-h] \
 [(--threads|-t) number_of_threads_per_processor] \
 [(--input|-i) name_of_master_input]
\end{verbatim}
If the master input file is the last argument, --input or -i
can be omitted. If the master input file has the name {\bf
  topmon.inp}, it can be left out as well.

%%%%%%%%%%%%%%%%%%%%%%%%%%%%%%%%%%%%%%%%%%%%%%%%%%%%%%%
\subsection{Basic Rules for Input Files}
For most simulations, there need to be two input files: {\bf
  topmon.inp} and {\bf fort.4}. {\bf topmon.inp} is designed
to contain parameters that are shared across all
simulations, while {\bf fort.4} includes all the other
parameters that can be different for each independent run.

Both input files are composed of two types of elements:
namelists and sections. Namelists are implemented using the
Fortran namelist mechanism, which start with
`` \&namelist\_name'' and end with `` /''. Pay attention to
the required, one-character space in front of ``\&'' and
``/''. The body of a namelist are lines with the format
``key=value(s)''. If key is an array variable, its elements
can be entered with spaces as delimiter.

Sections start with ``SECTION\_NAME'' and end with ``END
SECTION\_NAME'', the exact format of which is decided by the
programmer.

Comments can be added to assist readability. These are
anything after ``\#'', ``!'', or ``\%''. After stripping off
the comments, blank lines are ignored.

%%%%%%%%%%%%%%%%%%%%%%%%%%%%%%%%%%%%%%%%%%%%%%%%%%%%%%%
\section{Input File: \textbf{topmon.inp}}
\subsection{Namelist \textbf{io}}
\noindent{\bf file\_input:} Name of the individual input
file; default: \textbf{fort.4}.

\noindent{\bf file\_restart:} Name of the initial
configuration file; default: \textbf{fort.77}.

\noindent{\bf file\_struct:} Name of the molecule structure
file; default: \textbf{input\_struc.xyz}.

\noindent{\bf file\_run:} Name of the main output file;
default: \textbf{run1a.dat}.

\noindent{\bf file\_movie:} Name of the movie file; default:
\textbf{movie1a.dat}.

\noindent{\bf file\_solute:} Name of the movie file for
selected molecules; default: \textbf{fort.11}.

\noindent{\bf file\_traj:} Name of the trajectory file;
default: \textbf{fort.12}.

\noindent{\bf io\_output:} Set to 2 to print to a file. Set
to 6 to print to the screen.

\noindent{\bf run\_num, suffix:} The run number and suffix
are included as identifiers in the names of the output
files. In this manual, references to files such as {\bf
  config\#\#.dat} indicate a file whose name includes the
{\bf run\_num} and {\bf suffix}.

\noindent{\bf L\_movie\_xyz:} Yields a single movie file in
xyz format if TRUE.

\noindent{\bf L\_movie\_pdb:} Yields a single movie file in
pdb format if \textbf{TRUE}. Residue names are taken as first three
letters in the corresponding {\bf molecname} variable. If
molecule names are not specified, residue names will be
printed as "UND".

\subsection{Namelist \textbf{system}}
\noindent{\bf lnpt:} if TRUE then a NPT volume move is used
to equilibrate with a pressure bath (implies cubic
simulation boxes) else an NVT simulation is performed.

\noindent{\bf lgibbs:} if TRUE then a Gibbs-ensemble
simulation is performed (implies cubic simulation boxes).

\noindent{\bf lgrand:} if TRUE then simulation is performed
in the grand-canonical ensemble.

\noindent{\bf lanes:} if TRUE then simulation is performed in
the adiabatic nuclear and electronic sampling technique for
polarizable force fields.

\noindent{\bf lvirial:} if TRUE then one chain will be
simulated in each box independently and the second virial
coefficient will be calculated for their interactions at a
series of distances along the x-axis.

\noindent{\bf lmipsw:} if TRUE, then thermodynamic
integration is performed for the phases, fort.35 must be
supplied if so.

\noindent{\bf lexpee:} if TRUE, then expanded esnemble is
performed for the phases, {\bf fort.44} must be supplied if
so.

\noindent{\bf ldielect:} if TRUE then dielectric constant
will be calculated and LEWALD must be TRUE Correct only in
NVT ensemble.

\noindent{\bf lpbc:} if TRUE then periodic boundaries are used.

\noindent{\bf lpbcx:} if TRUE then periodic boundary in x-directions is used.

\noindent{\bf lpbcy:} if TRUE then periodic boundary in y-directions is used.

\noindent{\bf lpbcz:} if TRUE then periodic boundary in z-directions is used.

\noindent{\bf lfold:} if TRUE then coordinates are always folded into central box.

\noindent{\bf lijall:} if TRUE then all i-j interactions are
considered (no potential cut-off). Must set lcutcm and
ltailc to FALSE for lijall = TRUE *** check top of sumup.f
if lijall = TRUE and lchgall is FALSE.

\noindent{\bf lchgall:} if TRUE then all the electrostatic
interaction are considered.

\noindent{\bf lewald:} if TRUE then ewald-sum will be used to
calculate the electrostatic interactions.

\noindent{\bf lcutcm:} if TRUE then a cutoff of the centers
of mass will be used with a value of rcmu as calculated in
ctrmas.

\noindent{\bf ltailc:} if TRUE tail corrections are added
(WARNING: .lsami. in external.inc switches an intrinsic tail
correction on).

\noindent{\bf lshift:} if TRUE use truncated and shifted potentials.

\noindent{\bf ldual:} if TRUE then the external potential
during a CBMC growth will only go out to a radius of rcutin
and then will be corrected to the full rcut at the end. This
is Dual Cutoff Configurational-bias Monte Carlo (DC-CBMC).

\noindent{\bf L\_Coul\_CBMC:} If TRUE, uses the Coulombic
energy during the Boltzmann weight calculation. Normally
TRUE.

\noindent{\bf lneigh:} if TRUE the nearest neighbor list will
be used with a value of rcutnn specified in {\bf
  file\_input}.

\noindent{\bf lexzeo:} implicit rigid framework for zeolites
and metal-organic frameworks. Tabulated potential will be
used for the interactions with the rigid framework.
Parameters are read in subroutine SUZEO. See
\ref{zeolite_in} and \ref{zeolite} for more information.

\noindent{\bf lslit:} featureless, planar slit surface(s).

\noindent{\bf lgraphite:} x,y-dependent graphite surface.

\noindent{\bf lsami:} SAMI; LJ potential hard coded in {\bf susami}.

\noindent{\bf lmuir:} Langmuir monolayer; LJ potential hard coded in {\bf susami}.

\noindent{\bf lelect\_field:} external electric field.

\noindent{\bf lgaro:} if TRUE Feuston-Garofalini potential
will be used; parameters hard coded in
{\bf init\_garofalini}.

\noindent{\bf lionic:} disable the check for molecule neutrality

\noindent{\bf L\_Ewald\_Auto:} For the automatic assignment
of Ewald parameters. If TRUE, the code will generate {\bf
  kalp(ibox)} by dividing 3.2 by {\bf rcut}. This is the
standard method but always check for convergence of the
Ewald summation. Remember to set {\bf lewald} to TRUE to
turn on the Ewald summation.

\noindent{\bf lmixlb:} If TRUE, the Lorentz-Berthelot
combining rules for the Lennard-Jones potential will be
used. These are the rules used by the TraPPE force field and
what should be used by default.

\noindent{\bf lmixjo:} If TRUE, the Jorgensen combining
rules for the for the Lennard-Jones potential will be used.
These are the rules used by the OPLS force field.

\noindent{\bf L\_spline:} If torsional energy is tabulated,
use spline interpolation. Requires file {\bf fort.40}
running from -195 to 195 in degree steps (Extra 15 degrees
on each side required so that second derivatives are
reasonable for the degrees of interest).

\noindent{\bf L\_linear:} If torsional energy is tabulated,
use linear interpolation. Requires file {\bf fort.40}
running from -180 to 180 in 1/4 degree intervals.

\noindent{\bf L\_vib\_table:} Calculates vibrational
potential using linear interpolation. Requires file
{\bf fort.41} that starts with 0.0 (not 0.5). Must also specify
an equilibrium bond length and zero force constant.

\noindent{\bf L\_bend\_table:} Calculates 1-3 nonbonded
``bending'' using linear interpolation. Requires file {\bf
  fort.42} with distances in Angstrom starting at 0.0. The
tabulated bending is calculated using distances, not angles.
Additionally, in the {\bf topmon.inp} file you must specify
a potential with the same identifier as the potential in the
{\bf fort.42} file. The k-value you specify must be postive.

\noindent{\bf L\_elect\_table:} Calculate electrostatic
potential using linear interpolation. Currently, this is
designed to work with Greg Voth's
force-matched/coarse-grained potential, so each potential in
fort.44 is the same. The tabulated potentials are then
multiplied by the partial atomic charges within the code.

\subsection{Namelist \textbf{zeolite\_in}}
\label{zeolite_in}
Please see \ref{zeolite} for more information.

\noindent{\bf file\_zeocoord:} Name of the zeolite
coordinate file; default: \textbf{zeolite.cssr}.

\noindent{\bf dgr:} Grid spacing (\AA) for the tabulated
potential; default: 0.2. This affects the choice of
\textbf{rexcl} as explained in \ref{zeolite}.

\noindent{\bf file\_supercell} Name of zeolite pdb file.
If this variable is present, a pdb file will be output
with the supercell structure used in simulation.
This will happen during initialization (before beginning of
cycles).

\noindent{\bf file\_ztb:} Name of the zeolite tabulated
potential file; default: \textbf{zeolite.ztb}.

\noindent{\bf requiredPrecision:} When generating the
tabulated potential with \textbf{ltailcZeo} set to .true.,
this controls the precision at which point to stop include
more layers for the LJ interactions; default: 1.0E-2.

\noindent{\bf num\_points\_interpolation:} Number of points
used for potential interpolation; default: 5. A polynomial
interpolation of degree N requires N+1 points.

\noindent{\bf upperLimit:} A large value above which to
consider the interpolated energy to be infinity; default:
1.0E+5.

\noindent{\bf ltailcZeo:} Default to .true.. Has different
meanings for pre-tabulation and for interpolation. For
pre-tabulation .true. means successive layers of unit cells
need to be considered for LJ interactions until
\textbf{requiredPrecision} is achieved, which is to mimick
the use of the Ewald technique; otherwise \textbf{rcut} for
this simulation box will be used. If \textbf{ltailcZeo} is
set to .false. during pre-tabulation, you can, during
interpolation, optionally set it to .true. to use
(approximate) analytical tail corrections or set it to
.false. to use a spherically truncated potential. If
\textbf{ltailcZeo} is set to .true. during pre-tabulation,
then its setting during interpolation does not matter. To
use a shifted potential, set \textbf{ltailcZeo} to .false.
and \textbf{lshift} to .true. during pre-tabulation. The
treatment of electrostatic energies is controlled through
\textbf{lewald} similar to the way it is handled elsewhere.

\noindent{\bf ltestztb:} Set to .true. to test the accuracy
of the tabulated potential. The program will randomly select
\textbf{volume\_nsample} points, evaluate the energies both
explicitly and using interpolated values from the tabulated
potential, and compare the differences.

\noindent{\bf lpore\_volume:} Set to .true. to calculate the
pore volume as the adsorption second virial coefficient. See
O. Talu and A.L. Myers, "Molecular simulation of adsorption:
Gibbs dividing surface and comparison with experiment",
\textit{AICHE J.}, 47(5), 1160-1168 (2001). It also involves
energy calculations at randomly selected positions, so the
accuracy test and the calculation of pore volume can be done
together without much added computational cost.

\noindent{\bf volume\_probe:} Bead type (as defined in
\ref{atoms}) of the volume probe atom.

\noindent{\bf volume\_nsample:} Number of points to consider
for the accuracy test and the calculation of pore volume.

\noindent{\bf lsurface\_area:} Set to .true. to calculate
the geometric surface area using a Monte Carlo scheme. It
randomly generates \textbf{area\_nsample} points around each
framework atom in the unit cell on a sphere of radius
$\sigma_{\textrm{host,probe}}$ and calculates the overlap
with the rest of the framework atoms.

\noindent{\bf area\_probe:} Bead type (as defined in
\ref{atoms}) of the area probe atom.

\noindent{\bf area\_nsample:} Number of points around each
unit-cell framework atom to consider for the surface area
calculation.

\subsection{Section \textbf{ATOMS}}
\label{atoms}
This section specifies information for the bead types and
their interaction parameters. If there are duplicate
definitions for the same bead type, the last one takes
precedence.

\noindent{\bf i:} the bead type as used for {\bf unit}.

\noindent{\bf type:} the potential functional form for van
der Waals interaction. See \ref{vdW}. $-1$ switches on
tabulated potential for this interaction.

\noindent{\bf vdW parameters:} the parameters for the
selected potential functional form. See \ref{vdW}.

\noindent{\bf q:} the partial charge in $|e^{-1}|$.

\noindent{\bf mass:} the atomic weight.

\noindent{\bf chemid:} a 1--4 letter description. Useful for
visualizing the system in VMD.

\subsubsection{Supported Nonbonded Interaction Potentials}
\label{vdW}
\paragraph{Lennard-Jones 12-6}

\begin{equation}
U(r) = 4 \epsilon \left[\left(\frac{\sigma}{r}\right)^{12}
-\left(\frac{\sigma}{r}\right)^6 \right]
\end{equation}

{\bf type} is 1, {\bf vdW parameters} list the following in
order: $\epsilon$ (in Kelvin) and $\sigma$ (in~\AA).

\paragraph{Buckingham exponential-6}
\begin{equation}
U(r) = A \exp(-Br) - C/r^6
\end{equation}

{\bf type} is 2, {\bf vdW parameters} list the following in
order: $A$ (in Kelvin), $B$ (in~\AA$^{-1}$), and $C$ (in
Kelvin~\AA$^6$).

\paragraph{Mie}
\begin{equation}
U(r) = C \epsilon \left[\left(\frac{\sigma}{r}\right)^{n_0}
-\left(\frac{\sigma}{r}\right)^{n_1} \right] \\
\end{equation}
where $C = \frac{n_0}{n_0-n_1} \left(\frac{n_0}{n_1}\right)^{n_1/(n_0-n_1)}$

{\bf type} is 3, {\bf vdW parameters} list the following in
order: $\epsilon$ (in Kelvin), $\sigma$ (in~\AA), $n_0$, and
$n_1$.

\paragraph{MMFF94}
Merck Molecular Force Field (Thomas A. Halgren, \textit{J Am
  Chem Soc} 1992,\textbf{114}:7827-7843) is a buffered 14-7
potential.

\begin{equation}
U(r) = \epsilon \left( \frac{1.07}{r/r_0+0.07} \right)^7
\left[ \frac{1.12}{(r/r_0)^7+0.12}-2 \right]
\end{equation}

{\bf type} is 4, {\bf vdW parameters} list the following in
order: $\epsilon$ (in Kelvin) and $r_0$ (in~\AA). Note that
this potential does not support tail corrections. It will
simply be ignored without flagging an error, thus allowing
other potentials in the system to have tail corrections.

\paragraph{Lennard-Jones 9-6}
\begin{equation}
U(r) = 4 \epsilon \left[2\left(\frac{r_0}{r}\right)^9
-3\left(\frac{r_0}{r}\right)^6 \right]
\end{equation}

{\bf type} is 5, {\bf vdW parameters} list the following in order:
$\epsilon$ (in Kelvin) and $r_0$ (in~\AA).

\paragraph{Generalized Lennard-Jones}
This is NOT Mie potential using $r_0$ instead of $\sigma$!
See \textit{J Chem Phys} 2004,\textbf{120}:4994.
\begin{equation}
U(r) = \left\{
\begin{array}{rl}
4 \epsilon \left[\left(\frac{r_0}{r}\right)^{n_0}
-2\left(\frac{r_0}{r}\right)^{n_0/2} \right] & \text{if } r \leq r_0, \\
4 \epsilon \left[\left(\frac{r_0}{r}\right)^{2n_1}
-2\left(\frac{r_0}{r}\right)^{n_1} \right] & \text{if } r>r_0.
\end{array} \right.
\end{equation}

{\bf type} is 6, {\bf vdW parameters} list the following in
order: $\epsilon$ (in Kelvin), $r_0$ (in~\AA), $n_0$, and
$n_1$.

\paragraph{Lennard-Jones 12-6-8}
\begin{equation}
U(r) = \frac{A}{r^{12}} - \frac{B}{r^6} - \frac{C}{r^8}
\end{equation}

{\bf type} is 7, {\bf vdW parameters} list the following in
order: $A$ (in Kelvin~\AA$^{12}$), $B$ (in Kelvin~\AA$^6$),
and $C$ (in Kelvin~\AA$^8$).

{\bf type} is 8, {\bf DPD Parameters} list in the following order $\frac{a_ii}{2}$, $r_c$ in $\AA$, follwed by 0.0 1.0 and an atom name. Soft repulsive potential for non-Polymer Dissipative Particle Dynamics

\begin{equation}
U\left(r_{ij}\right) = \frac{a_{ij}}{2}\left(1-\frac{r_{ij}}{r_c}\right)^2
\end{equation}
Note that in the file {\bf fort.4} the rcut also needs to be set to the same $r_c$ as seen in this file. An example line would read: \\ \indent {\tt  401 8 12.5 10.000 0.0 1.0 A \# DPD 1} \\ for a system where the cut off is 10 $\AA$ and $a_{ij} = 25$. \\ see section \ref{sec:dpdunitconversions}

\subsection{Section \textbf{NONBOND}}
\label{nonbond}
This section specifies the interaction parameters between
unlike beads, if there is no combining rules for the
potential form or if there is exceptions to the default
combining rules. {\bf type} and {\bf vdW parameters} have
the same meaning as in \ref{atoms}. See \ref{vdW}.

\noindent{\bf i, j:} the two bead types as used for {\bf
  unit}.

\noindent{\bf type:} the potential functional form for van
der Waals interaction. $-1$ switches on tabulated potential
for this interaction.

\noindent{\bf vdW parameters:} the parameters for the
selected potential functional form.


\subsection{Section \textbf{BONDS}}
\label{bonds}

\noindent{\bf i:} specifies the type as used for {\bf
  stretching}.

\noindent{\bf type:} the potential functional form for
stretching. Currently can only be 1, indicating harmonic
potential.

\noindent{\bf brvib:} equilibrium bond lengths in~\AA.

\noindent{\bf brvibk:} force constant, $k/2$, in
Kelvin/\AA$^2$. A value of zero indicates fixed bond length.

\noindent{\bf Optional variables:}
 These variables are optional and should be included when 
you are running a simulation using CBMC and flexible bonds.
Default values if not included are 0.0 and 2.0 meaning the 
bond can regrow to $\pm$ 100\% of their equilibrium bond
length. These values are inefficient but likely to select
the entire range of possible values. Below is described
a method by which these values can be selected. 
 
\noindent{\bf minimumRegrow:} fraction describing shortest 
allowable bond length during a CBMC regrowth when allowing 
for flexible bonds. Expressed as $L_{min}$/brvib.

\noindent{\bf maximumRegrow:} fraction describing longest 
allowable bond length during a CBMC regrowth when allowing 
for flexible bonds. Exressed as $L_{max}$/brvib.

Extreme care should be taken when selecting the maximum and 
minimum values for regrowth of flexible bonds when using CBMC 
move types. If you have too wide of a range then your simulation
will be very inefficient as it will sample too many bond lengths 
that it is likely to reject. If your range is too narrow then 
your simulation will not produce accurate results. You should 
test your values by running shorter simulations to ensure that
you are not arbitrarily cutting short your distribution. Bond
lengths will generally fall with in 4 standard deviations of 
your equilibrium bond length where the standard devition
$\sigma=\sqrt{(\beta k)^{-1}}$ where $k$ is the force constant
and $\beta=k_B T$. Plots of the bond length distribution for 
each bond type can be used to determine if this value is 
reasonable for you.  

\subsection{Section \textbf{ANGLES}}
\label{angles}
\noindent{\bf i:} specifies the type as used for {\bf
  bending}.

\noindent{\bf type:} the potential functional form for
bending. Currently can only be 1, indicating harmonic
potential.

\noindent{\bf brben:} equilibrium bending angle in degrees.

\noindent{\bf brvibk:} force constant, $k/2$, in
Kelvin/rad$^2$. A value of zero indicates fixed bending
angle, and negative values indicate freely-joint beads.

\subsection{Section \textbf{DIHEDRALS}}
\label{dihedrals}
\noindent{\bf i:} specifies the type as used for {\bf
  torsion}.

\noindent{\bf type:} the potential functional form for
torsion. See \ref{torsion}. $-1$ switches on tabulated
potential for this interaction, and a value of zero means no
torsional energy (the torsion is just for setting up
intramolecular exclusion).

\noindent{\bf vtts:} the parameters for the selected
potential functional form. See \ref{torsion}.

\subsubsection{Supported Dihedral Potentials}
\label{torsion}
{\bf NOTE} that in this section angle denoted as $\psi$ uses
polymer convention (trans is 0 deg), while angle denoted as
$\phi$ uses protein convention (trans is 180 deg).

\paragraph{OPLS (3 terms)}
\begin{equation}
V(\phi) = vtt0 + vtt1\cdot[1+\cos(\phi)] + vtt2\cdot[1-\cos(2\phi)]
+ vtt3\cdot[1+\cos(3\phi)]
\end{equation}
{\bf type} is 1.

\paragraph{Ryckaert-Bellemans (9 terms)}
\begin{equation}
V(\psi) = vtt0 + vtt1\cdot\cos(\psi) + vtt2\cdot\cos^2(\psi) + \dots + vtt9\cdot\cos^9(\psi)
\end{equation}
{\bf type} is 2.

\paragraph{Periodic}
\begin{equation}
V(\phi) = vtt0\cdot[1+\cos(vtt1\cdot\phi-vtt2)]
\end{equation}
{\bf type} is 3.

\paragraph{Harmonic}
\begin{equation}
V(\psi) = vtt0\cdot(\psi-vtt1)^2
\end{equation}
{\bf type} is 4.

\paragraph{OPLS (4 terms)}
\begin{equation}
V(\phi) = vtt0 + vtt1\cdot[1+\cos(\phi)] + vtt2\cdot[1-\cos(2\phi)]
+ vtt3\cdot[1+\cos(3\phi)] + vtt4\cdot[1-\cos(4\phi)]
\end{equation}
{\bf type} is 5.

\paragraph{Fourier expansion (9 cosine terms)}
\begin{equation}
V(\phi) = vtt0 + vtt1\cdot\cos(\phi) + vtt2\cdot\cos(2\phi) + \dots
+ vtt9\cdot\cos(9\phi)
\end{equation}
{\bf type} is 6.

\paragraph{Ryckaert-Bellemans (3 terms)}
\begin{equation}
V(\psi) = vtt0 + vtt1\cdot\cos(\psi) + vtt2\cdot\cos^2(\psi) + vtt3\cdot\cos^3(\psi)
\end{equation}
{\bf type} is 7.

\paragraph{Ryckaert-Bellemans (4 terms)}
\begin{equation}
V(\psi) = vtt0 + vtt1\cdot\cos(\psi) + vtt2\cdot\cos^2(\psi) + vtt3\cdot\cos^3(\psi) + vtt4\cdot\cos^4(\psi)
\end{equation}
{\bf type} is 8.


%%%%%%%%%%%%%%%%%%%%%%%%%%%%%%%%%%%%%%%%%%%%%%%%%%%%%%%
\section{Input File: \textbf{fort.4}}
This is the main input file for a simulation. It contains
information about the system to be simulated, including the
number and type of molecules, how these molecules are to be
modeled and what type of moves will be utilized. An error
from the {\bf readdat.f} file often indicates a problem in
the {\bf fort.4} input file. Note that most variables, even
if they are not used in a particular simulation, must be
present in {\bf fort.4} at all times; however, there are a
few variables that must be removed depending on the setting
of other variables. This manual specifies those variables
that are sometimes removed; if not specified, assume that a
particular variable must be included in {\bf fort.4}.

\subsection{Namelist \textbf{mc\_shared}}
\noindent{\bf seed:} Initializes the random number generator
with an integer seed value. If the seed stays the same, the
program will always generate the same sequence of
pseudo-random numbers, which is useful for debugging. Change
the seed to generate a different sequence. Must be positive
for parallel runs.

\noindent{\bf nbox:} The number of simulation boxes. Note
that some variables, including {\bf boxlx, boxly, boxlz,
  lsolid, lrect, kalp, rcut} and {\bf rcutnn}, must be
specified for each box.

\noindent{\bf nmolty:} The number of different molecule
types in the simulation.

\noindent{\bf nchain:} The total number of molecules in the simulation.

\noindent{\bf nmax:} In GCMC, the maximum number of molecules that
will be allowed. If not provided, nmax defaults to nchain + 2

\noindent{\bf nstep:} The number of Monte Carlo cycles
(steps) to run the simulation.

\noindent{\bf time\_limit:} Time in seconds specifying for
how long the code should run. It will exit either after {\bf nstep}
or after {\bf time\_limit} seconds, whichever happens first. This
is useful when running on HPCs to fully utilize alloted walltimes. 
The limit has a (120s + average time per cycle) grace period 
for the final sumup and other finishing touches. 
Should sumup take longer than that {\bf time\_limit} 
should be adjusted correspondingly. The variable is
optional and the default value is 0.

\noindent{\bf lstop:} If TRUE, {\bf nstep} refers to MC
steps. If FALSE, {\bf nstep} refers to cycles. This
statement controls all references to ``cycles'' or ``steps''
and should be FALSE unless debugging or GCMC.

\noindent{\bf iratio:} The maximum displacements for atom
translation, molecular translation and rotation are updated
every {\bf iratio} cycles. Smaller values (approximately
250-500) are acceptable for equilibration. For production,
{\bf iratio} must be greater than {\bf nstep} to satisfy
microscopic reversibility.

\noindent{\bf rmin:} The minimum cutoff for atom-atom
distances. Any move that would bring atoms closer than this
distance will be automatically rejected. In theory {\bf
  rmin} can be set to any value as long as $g(r)$ is zero up
to that distance. A smaller value, even 0~{\AA}, may be used
for equilibration than production. A good value for a system
that contains hydrogen bonds is 1.2~\AA. A good value for a
system without hydrogen bonding is $0.7\sigma$.

\noindent{\bf softcut:} The upper bound on numbers of which
to take the negative exponential. Used to rule out events
with low probability to avoid numerical overflow. 100.0 is a
reasonable value for double precision builds. 38 is the
maximum value that can be used for single precision builds.

\noindent{\bf checkpoint\_interval:} Time in seconds to
write out {\bf save-config.i} and {\bf save-stats.i}, which
are a snapshot of current system configuration with the same
format as {\bf file\_restart} and a snapshot of all counters
and accumulators. Useful if a simulation ends unexpectedly
so that progress up to that point can be kept, in which case
you can either rename {\bf save-config.i} to be the next
starting configuration file, {\bf file\_restart}, and start
the next simulation normally (thus leaving incomplete {\bf
  file\_run}, {\bf file\_movie}, etc., which is fine for
equilibration), or rename {\bf save-config.i} as {\bf
  save-config} and {\bf save-stats.i} as {\bf save-stats} to
use checkpointing.

\noindent{\bf checkpoint\_copies:} How many copies of {\bf save-config} to keep.

\noindent{\bf use\_checkpoint:} If TRUE, continue a
simulation that is interrupted, using {\bf save-config} and
{\bf save-stats}. Files are appended as if they are never
interrupted. You should NOT modify any of the unfinished
output files in ANY way nor change any input parameters
other than setting \textbf{use\_checkpoint} to TRUE.

\noindent{\bf linit:} If TRUE, the simulation will be
initialized. This variable should be TRUE for the initial
run and FALSE for all subsequent runs. Note that {\bf
  file\_restart} is required if {\bf linit} is FALSE.

\noindent{\bf lreadq:} If TRUE, the bead charge values will
be read from the {\bf file\_restart}. If FALSE, the charges
will be taken from force field definitions. This should be
used only for fluctuating charge models.

\noindent{\bf N\_add:} Number of chains to add to (positive)
or to remove from (negative) the system.

\noindent{\bf box2add:} The box which chains will be
added to/removed from.

\noindent{\bf moltyp2add:} The molecule type of the
added/removed chains.

\subsection{Namelist \textbf{analysis}}
\noindent{\bf iprint:} The simulation status will be printed
every {\bf iprint} cycles. Can be used to track the progress
of a simulation as it is running. A useful value yields
10-20 outputs per simulation.

\noindent{\bf imv:} The system configuration is sent to the
movie file, {\bf file\_movie}, every {\bf imv} cycles. As
this is used for subsequent analysis, it is important to
have reasonably statistically independent results. Larger
values of {\bf imv} (for example, greater than 500) will
yield results that are less correlated (more statistically
independent).

\noindent{\bf iblock:} The number of cycles to be
independently averaged when estimating statistical
uncertainties for $p$, $\rho$ or mole fraction, etc. At
least 1000 cycles per block is a good value. A constant
drift in the block average of a given property (for
instance, steadily decreasing energy values) indicates that
the system is not yet equilibrated.

\noindent{\bf iratp:} The pressure is calculated every {\bf
  iratp} cycles. As calculating the pressure is somewhat
computationally intense, this typically should be set to 5
or greater. A good value is perhaps 10.

\noindent{\bf idiele:} For dielectric constant calculations,
the system dipole is calculated every {\bf idiele} cycles
and the $x$, $y$ and $z$ components are written to {\bf
  fort.27}. The dielectric constant needs to be calculated
separately using {\bf fort.27} (see description of {\bf
  fort.27} in section \ref{output}). For single-component
$NpT$ simulations, {\bf idiele} also controls the frequency
of the output volume and energy statistics as output in {\bf
  fort.14} through {\bf fort.19}.

\noindent{\bf iheatcapacity:} If {\bf iheatcapacity} is less
than {\bf nstep} and {\bf lnpt} is FALSE, then {\bf fort.55}
contains the following averages: $<E>$ and $<E^2>$. If {\bf
  iheatcapacity} is less than {\bf nstep} and {\bf lnpt} is
TRUE, then {\bf fort.56} contains the following averages:
$<H>$ and $<H^2>$.

\noindent{\bf ianalyze:} The analysis subroutine is called
every {\bf inanalyze} cycles. Could be a number smaller than
{\bf imv} to yield better statistics.

\noindent{\bf nbin:} The number of bins for the radial
distribution function calculation. A value of 200 should be
adequate for most purposes.

\noindent{\bf lrdf:} If TRUE, the radial distribution function (rdf) will be calculated.

\noindent{\bf lintra:} If TRUE, intramolecular distances will be included in the rdf.

\noindent{\bf lstretch:} If TRUE, the bond stretch distribution will be calculated.

\noindent{\bf lgvst:} If TRUE, \textit{gauche} versus
\textit{trans} statistics will be calculated.

\noindent{\bf lbend:} If TRUE, the bending distribution will be analyzed.

\noindent{\bf lete:} If TRUE, the end-to-end distribution of
flexible chains will be calculated. Flexible chains should
be numbered so that the first bead is at one end of the
chain and the last bead is at the other end.

\noindent{\bf lrhoz:} If TRUE, Z profiles (for solid slabs
or self-assembled monolayers) will be calculated. The solid
part must be at the center of the simulation box.

\noindent{\bf bin\_width:} Specifies the bin width for the
end-to-end distribution and density profile. A value of 0.2
is adequate for most purposes.

\noindent{\bf lucall:} If TRUE, the chemical potential is
calculated independently. Usually FALSE since the chemical
potential is determined automatically by the swap move.

\noindent{\bf ucheck:} If {\bf lucall} is TRUE, {\bf ucheck}
indicates for which molecule types the chemical potential
should be calculated. A value of 0 means the chemical
potential will not be calculated. A value greater than 0
means the chemical potential will be calculated. There must
be {\bf nmolty} values.

\noindent{\bf nvirial:} If {\bf lvirial} is TRUE, the second
virial coefficient will be calculated for a molecule. {\bf
  nvirial} determines the number of configurations to
consider at each step. This is a specialized simulation and
not for the general case.

\noindent{\bf startvir, stepvir:} If the second virial
coefficient is being calculated, these determine the
starting center of mass distance and step size, in {\AA}.

\subsection{Namelist \textbf{external\_field}}
\noindent{\bf Elect\_field:} The electric field strength, in
units of V/\AA, for each simulation box. The electric field
is applied in the $z$-direction. Note that {\bf
  lelect\_field} must be set to TRUE to enable external
electric field.

\subsection{Monte Carlo Move Probabilities}
The simulation will carry out {\bf ncycles} cycles.
Available move types include volume, swatch, swap, CBMC,
fluctuating charge, (new) expanded ensemble, atom
translation, translation, and rotation moves; the
corresponding probability for each move type is given by
{\bf pmvol, pmswat, pmswap, pmcb, pmflcq, pmexpc (pmexpc1),
  pm\_atom\_tra}, {\bf pmtra}, and $1.0$, respectively. Move
probabilities are given cumulatively. For example, if ${\bf
  pmvol} = 0.10$ and ${\bf pmswat} = 0.25$, there is a
$10\%$ chance of choosing a volume move and a $25\%-10\% =
15\%$ chance of choosing a swatch move. Since the move
probabilities are given cumulatively, the probability for a
rotation move is simply $1 - {\bf pmtra}$. Sub-move
probabilities, such as on which box to perform a volume
move, are also cumulative.

\subsubsection{Namelist \textbf{mc\_volume}}
\noindent{\bf tavol:} Specifies the target acceptance rate
for volume moves. We used to think reasonable values for volume equilibration
and full equilibration might be 0.4 and 0.5, respectively.
Ilja (9/24/15):"Please take a look at 
Mountain, R. D.; Thirumalai, D. Physica A: Statistical Mechanics and its Applications
1994, 210, 453–460.
Although MCCCS-MN is much more complex, I think that our
recommended target acceptance rate for translation, rotation,
and volume moves should be about 30\%.  [50\% is almost certainly too high.]"

\noindent{\bf iratv:} The maximum volume displacement will
be adjusted every {\bf iratv} cycles. {\bf iratv} is
analogous to {\bf iratio}, which controls how often the
maximum translation and rotation displacements are adjusted.
As mentioned for {\bf iratio}, smaller values such as
250-500 are acceptable for equilibration. Note that {\bf
  iratv} must be greater than {\bf nstep} for production to
satisfy microscopic reversibility.

\noindent{\bf pmvol:} Specifies the probability to perform a
volume move. Typically set to 2 divided by {\bf nchain} to
have approximately one accepted volume move per MC cycle.

\noindent{\bf pmvlmt:} The probability to perform a volume
move on each box. Must be one probability for each
simulation box.

\noindent{\bf nvolb:} For the $NVT$-Gibbs ensemble, the
number of pairs of boxes on which to perform volume exchange
moves.

\noindent{\bf pmvolb:} The probability to perform a volume
exchange on each pair of boxes in an $NVT$-Gibbs ensemble.
Must be {\bf nvolb} values.

\noindent{\bf box5, box6:} Specifies a pair of boxes for
volume exchange. Must be {\bf nvolb} values for each. Not
used if the simulation is in the $NVT$, $NpT$, or
$NpT$-Gibbs ensemble.

\noindent{\bf pmvolx, pmvoly:} These values are used when at
least one of the simulation boxes is non-orthorhombic. These
values set the (cumulative) probabilities to do moves in the
$x$ and $y$ directions, with the $z$ probability being 1 -
{\bf pmvoly}.

\noindent{\bf rmvolume:} This value specifies the initial
volume displacement. For an $NpT$(-Gibbs) simulation a value
of $10^3$ is appropriate, while $10^{-3}$ is a good choice
for an $NVT$-Gibbs simulation. The difference is due to the
fact that in $NVT$-Gibbs simulations, volume moves are
performed in logarithmic space.

\noindent{\bf allow\_cutoff\_failure:} controls how volume
move failures, the ones that will result in box lengths
smaller than twice the cutoff, are handled: -1 = fetal error
and program exits; 0 = simply rejects the move, which is
equivalent to modifying the lower limit of integration of
the partition function; 1 = allows the move and adjusts
cutoff to be half of the new box lengths (will be restored
if possible); 2 = allows the move but does not adjust the
cutoff, which could be problematic because this results in a
lower density in the cutoff radius (due to periodic boundary
conditions)

\subsubsection{Namelist \textbf{mc\_swatch}}
\noindent{\bf pmswat:} Specifies the probability to perform
a swatch (CBMC particle identity switch) move.

\noindent{\bf nswaty:} The number of pairs of molecule types
to perform swatch moves on.

\noindent{\bf pmsatc:} Once a swatch move is being
performed, this specifies the probability to perform it on a
particular pair. Must be {\textbf nswaty} values.

\subsubsection{Section \textbf{MC\_SWATCH}}
Swatch move can be done by either regular CBMC (two molecules 
differ only at one or two ends) or SAFE-CBMC (two molecules 
also differ in the internal part). For each of the \textbf{nswaty} 
pairs, including the following, if it is regular CBMC swatch pair:

\noindent{\bf nswatb:} Specifies the two molecule types for this swatch pair.

\noindent{\bf nsampos:} Specifies the number of beads that
can be left in the same positions.

\noindent{\bf 2xncut:} Number of CBMC growth sites the
structure given by {\bf nsampos} has to grow back the whole
molecule.

\noindent{\bf gswatc 2x(ifrom, iprev):} specifies the bead
to grow from and the previous bead for both molecule types.
These should almost always be beads listed in {\bf splist},
unless {\bf ifrom} is the first/last bead, in which case
{\bf iprev} can be set to 0.

\noindent{\bf splist:} Specifies the beads in the two
molecules to be kept in the same position. Must be {\bf
  nsampos:} pairs.

\noindent{\bf nswtcb:} Number of box pairs to swatch molecules.

\noindent{\bf pmswtcb:} Probabilities to pick a particular
box pair for a swatch move. Must be {\bf nswtcb} values.

\noindent{\bf box\_numbers:} Specifies the two boxes in each
of the {\bf nswtcb} pairs. Note that two boxes can be the
same, in which case the move is performed inside a single
box. \\

\noindent {\bf Example \#1}\hfill\break
Swatch move between ethane and propane, between box 1 and 2.

\begin{verbatim}
MC_SWATCH
! moltyp1<->moltyp2 nsampos 2xncut  1. Ethane (1-2) <--> Propane (1-2-3)
  1           2       2      1 1
! gswatc 2x(ifrom, iprev)
  2 1 2 1
! splist
   1 1
   2 2
! nswtcb pmswtcb
1  1.00
! box numbers
1 2
END MC_SWATCH
\end{verbatim}

\noindent In this example, suppose that we are swatching between ethane and propane, keeping the first two beads of ethane/propane at the same place. {\bf nsampos} and {\bf splist} are set in this way so that bead no. 1 and 2 are kept in the same position. {\bf 2xncut} is 1 since only one growth direction is needed. To grow the additional bead, we are growing from bead no. 2, and thus {\bf ifrom} is 2 and {\bf iprev} is 1. The second set of (ifrom, iprev) represents the corresponding beads on propane. \\

\noindent To use swatch between two molecules that differ in the internal part, there are additional information that need to be stated:

\noindent{\bf S}: The presence of S in the beginning of the line indicates that this swatch move contains SAFE-CBMC part.

\noindent{\bf 2xncut-safe}: Similar to {\bf 2xncut}, the number of interior segments that need to be regrown to swatch between two molecules. Note that one segment may contain many beads.

\noindent{\bf gswatcsafe 2x(ifrom, iprev, index)}: Specify where SAFE-CBMC starts ({\bf ifrom}), the previous bead of {\bf ifrom} ({\bf iprev}) and the number of {\bf backbone} beads to regrow. Note that if there is no regular CBMC part ({\bf 2xncut} = 0), the line containing {\bf gswatc} must be deleted!  

\noindent {\bf IMPORTANT}: Additionally, set one {\bf pmfix} in the {\bf mc\_cbmc} section to be larger than zero so that the fort.23 SAFE-CBMC guiding probability input file can be properly read. You can set it to be a small number to avoid doing SAFE-CBMC. Only one fort.23 file is needed. Moreover, remember to set {\bf iring} correctly (usually the same as {\bf nunit}) to enable SAFE-CBMC functionality. \\

\noindent {\bf Example \#2}\hfill\break
Swatch move between 3-methylpentane and pentane, between box 1 and 2.

\begin{verbatim}
MC_SWATCH
! moltyp1<->moltyp2 nsampos 2xncut 2xncut-safe  
! 3-methylpentane (1-2-3(6)-4-5) <--> Pentane (1-2-3-4-5)
S  1           2       3    0 0  1 1
! gswatcsafe 2x(ifrom, iprev, index)
  2 1 2 2 1 2
! splist
   1 1
   2 2
   5 5
! nswtcb pmswtcb
1  1.00
! box numbers
1 2
END MC_SWATCH
\end{verbatim}

\noindent In this case, beads are numbered so that the linear part is from 1 to 5, with methyl branch on 3-methylpentane being bead no. 6. We are doing swatch with SAFE-CBMC, keeping bead no. 1, 2 and 5 in the same place and regrowing bead no. 3, 4 and 6. {\bf 2xncut} are 0 because no regular CBMC part is needed, and {\bf 2xncut-safe} are 1 since one interior segment (containing 2 or 3 beads) is to be regrown. Since we are regrowing from bead no. 3, {\bf ifrom} is 2 and {\bf iprev} is 1. Two backbone beads are to be regrown so that {\bf index} is 2.


\subsubsection{Namelist \textbf{mc\_swap}}
\noindent{\bf pmswap:} Specifies the probability to perform
a swap move in which a molecule is removed from one
simulation box and placed in another box by growing it with
CBMC. It may be efficient to have approximately one accepted
swap move every 1-10 MC cycles.

\noindent{\bf pmswmt:} This specifies the probability to
swap a molecule of a particular type. There must be {\bf
  nmolty} values.

\subsubsection{Section \textbf{MC\_SWAP}}
For each of the \textbf{nmolty} molecules, including the following:

\noindent{\bf nswapb:} The number of box pairs to perform a
swap move.

\noindent{\bf pmswapb:} The probability to perform a swap
move on a particular pair of boxes. Must be {\bf nswapb}
values.

\noindent{\bf box1, box2:} Specifies the two boxes for each
swap pair. If {\bf box1} and {\bf box2} are the same, an
AVBMC move will be performed, in which case
\textbf{avbmc\_version} needs to be set.

\subsubsection{Namelist \textbf{mc\_cbmc}}
\paragraph{Basic CBMC Variables}

\noindent{\bf rcutin:} Specifies the inner cutoff for
dual-cutoff CBMC. During CBMC growths, only intermolecular
interactions between atoms within {\bf rcutin} of one
another are calculated. Smaller values of {\bf rcutin} will
require less computer time while larger values will have
higher acceptance rates. Typical values are between 5 and
9~\AA.

\noindent{\bf pmcb:} Specifies the probability to perform a
configurational-bias Monte Carlo (CBMC) move, which involves
regrowing a random segment of a molecule in a new
configuration in a position near its original position.
Typically approximately one-third of the moves in a flexible
system are CBMC moves. If dual-cutoff CBMC is desired,
remember to set {\bf ldual} to TRUE.

\noindent{\bf pmcbmt:} When a CBMC move is being performed,
{\bf pmcbmt} specifies the probability to choose a particular
molecule type. There must be {\bf nmolty} values.

\noindent{\bf pmall:} The probability that a particular
molecule type should try to regrow itself in its entirety
when a CBMC move is performed. Usually set to zero. Must be
{\bf nmolty} values.

\noindent{\bf nchoi1:} Specifies the number of randomly
chosen positions to consider during a trial insertion of the
first bead during a CBMC swap move. 32 is a reasonable
number. Increasing {\bf nchoi1} increases the computer time
but also increases the acceptance rate of the swap move.
There must be {\bf nmolty} values.

\noindent{\bf nchoi:} Specifies the number of randomly
chosen positions to consider during all CBMC regrowths. A
larger value takes longer but increases the acceptance rate.
16 is a reasonable value. There must be {\bf nmolty} values.
Note that {\bf nchoi} controls both swap and CBMC moves,
while {\bf nchoi1} controls only swap moves.

\noindent{\bf nchoir:} Specifies the number of rotations to
consider for the swap move of a rigid molecule. 16 is a
reasonable value. Only used if {\bf lrigid} is TRUE for this
molecule. There must be {\bf nmolty} values.

\noindent{\bf nchoih:} Specifies the number of explicit
hydrogen states to consider during a CBMC regrowth. The
value only changes the rotation of the methyl group. Unused
for other molecules. There must be {\bf nmolty} values.

\noindent{\bf nchoitor:} Specifies the number of torsional
angles to consider during a CBMC regrowth. 100 is a good
choice. Larger takes longer but does not increase acceptance
rates significantly. There must be {\bf nmolty} values.

\noindent{\bf nchbna, nchbnb:} Specifies the number of bend
angles to consider during a CBMC regrowth. 1000 is a good
choice. There must be \textbf{nmolty} values for each. {\bf
  nchbnb} is used for branched molecules for the angles
between two branches.

\noindent{\bf icbdir:} Specifies whether a CBMC growth
should go in only one direction. The usual value, 0,
indicates no preference. 1 indicates that CBMC growths are
considered only for increasing unit numbers. There must be
{\bf nmolty} values.

\noindent{\bf icbsta:} Specifies the starting unit for a
CBMC growth. If a consistent starting point is desired for a
CBMC move, {\bf icbsta} will specify the unit number. For
example, +5 indicates that CBMC growths should always start
at bead number 5. A value of -5 indicates that the growth
should start at a random position between bead 5 and the
highest numbered bead (this is useful when one end of the
chain is rigid and tethered to a surface). {\bf icbsta} is
usually set to 0, which means there is no preferred starting
point. If not set to 0, make sure that the absolute value of
{\bf icbsta} + {\bf maxcbmc} is equal to {\bf nunit} + 1 for
consistency. There must be {\bf nmolty} values of {\bf
  icbsta}.

\paragraph{AVBMC Variables}
\label{avbmc_vars}

The variables in this section will be used if AVBMC moves
are to be performed (that is, {\bf avbmc\_version} is 1, 2,
or 3 and intrabox swap moves are specified).

\noindent{\bf rbsmax, rbsmin:} Specifies the maximum and
minimum radii of the ``in'' region during an AVBMC move.
These values are dependent on the type of molecule. Typical
values for an alcohol may be 5~\AA~and 3~\AA, respectively.

\noindent{\bf avbmc\_version:} Can be 1, 2, or 3.

\noindent{\bf pmbias:} This value specifies to probability
with which an ``out to in'' move is attempted. The
probability of an ``in to out'' move is then given by $1 -
{\bf pmbias}$. There must be {\bf nmolty} values.

\noindent{\bf pmbsmt:} This variable gives the probability
with which the target site is a particular molecule type of
AVBMC 2 and 3 moves. There must be {\bf nmolty} values.

\noindent{\bf pmbias2:} This value gives the probability in
an AVBMC 3 move that a molecule is taken from the ``out''
region and moved to a target side. If the molecule is not
taken from the ``out'' region, it is removed from another
cluster. There must be {\bf nmolty} values.

\paragraph{SAFE-CBMC Variables}

\noindent{\bf pmfix:} Once a molecule has been chosen for a
CBMC move, {\bf pmfix} specifies the probability to perform
a SAFE-CBMC move. Usually 0 unless the molecule is large or
contains flexible rings. Must be {\bf nmolty} values.
SAFE-CBMC moves require the {\bf fort.23} input file, which
can be generated via a presimulation (see {\bf lpresim}). 
Currently, only one {\bf fort.23} file is needed for 
all the molecules.
Must be larger than zero if any swatch pair uses SAFE-CBMC.

\noindent{\bf lrig:} If TRUE, rigid segments should be grown
from non-rigid segments during SAFE-CBMC. The default
setting is FALSE.

\noindent{\bf lpresim:} If TRUE, runs a presimulation to
generate the SAFE-CBMC probabilities. There must be only one
particle in the simulation and {\bf pmfix} must be set to 0.
If SAFE-CBMC probabilities are required for multiple
particle types, run a presimulation for each particle type
and combine the probabilities in the {\bf fort.23} file
before beginning the simulation with multiple types of
particles.

\noindent{\bf iupdatefix:} Specifies in cycles how often the
SAFE-CBMC probabilities should be updated.

\subsubsection{Section \textbf{SAFE\_CBMC}}
\noindent{\bf nrig:} Only used if {\bf lrig} is TRUE. If
{\bf nrig} is less than or equal to 0, the site from which
to grow the rigid sites will be chosen randomly. If
{\bf nrig} is greater than 0, {\bf nrig} specifies the
number of specific points to be kept rigid during a
SAFE-CBMC growth.

\noindent{\bf irig:} Only used if {\bf lrig} is TRUE and if
{\bf nrig} is greater than 0. Specifies the site from which
the rigid part will be grown. There must be \textbf{nrig}
values.

\noindent{\bf frig:} Only included if {\bf lrig} is TRUE and
if {\bf nrig} is greater than 0. Specifies the site prior to
{\bf irig}. The site denoted by {\bf frig} is not kept
rigid. Must be {\bf nrig} pairs of {\bf irig} and {\bf
  frig}.

\noindent{\bf nrigmin:} Only included if {\bf lrig} is TRUE
and if {\bf nrig} is less than or equal to 0. This variable
denotes the minimum amount of the chain to keep rigid.

\noindent{\bf nrigmax:} Only included if {\bf lrig} is TRUE
and if {\bf nrig} is less than or equal to 0. This variable
denotes the maximum amount of the chain to keep rigid.

\subsubsection{Namelist \textbf{mc\_flucq}}

The variables in this section pertain to polarizable models that use
fluctuating charges. Independent charge updates (FQ moves) can
be performed by setting {\bf pmflcq} to a nonzero value. However, it
is usually best to use ANES-MC to sample electrostatic
degrees of freedom in systems with polarizable models.
ANES-MC, which can be turned on by setting {\bf lanes}=TRUE in topmon.inp,
uses a combined move of both nuclear position changes (trans, rot, etc.) 
and charge updates. For full details of the algorithm, please read 
Chen \& Siepmann, TCA, 103, 87 (1999). To run in the Gibbs ensemble,
also read Chen, Potoff \& Siepmann, JPCB, 104, 2378 (2000).

\noindent{\bf taflcq:} Specifies the target acceptance ratio
for fluctuating charge moves. Default: 0.5.

\noindent{\bf fqtemp:} The fluctuating charge temperature,
in Kelvin, for simulations using the ANES-MC or polarizable
models. Default: 5.0 K. For the special case where all the FQ moves 
are accepted (making the acceptance of the ANES-MC combined
move depend only on the system temperature, a set-up
equivalent to $T_{elec}$ = infinity), set {\bf fqtemp} to a value less than 
0 (e.g., $-1.0$).

\noindent{\bf rmflucq:} Specifies the maximum fluctuating
charge displacement. Default: 0.1.

\noindent{\bf pmflcq:} Specifies the probability to perform
an indepenent fluctuating charge move. Note: this probability pertains to
independent charge updates only--that is, charge updates done
{\sl without} ANES-MC. Running ANES-MC will automatically 
cause charge updates to be performed as part of the translation, 
rotation, swap, swatch, or volume moves and
does not require this probability to be set.

\noindent{\bf pmfqmt:} Specifies the probability to
carry out fluctuating charge moves on a particular molecule type,
used by both ANES-MC and independent charge updates. 
Must be {\bf nmolty} values.

\noindent{\bf lflucq:} If TRUE, the charge(s) on that
molecule type may fluctuate. Default: FALSE. 
Must be {\bf nmolty} values.

\noindent{\bf lqtrans:} If TRUE, {\sl inter}molecular charge
transfers for that molecule's type are allowed. ANES-MC can run with
two types of charge constraints: {\sl system} neutrality or {\sl molecule} neutrality.
Setting lqtrans to TRUE applies the system neutrality constraint. Setting
lqtrans to FALSE applies the molecule neutrality constraint (only 
{\sl intra}molecular charge transfers are allowed).
Default: FALSE.  Must be {\bf nmolty} values.

\noindent{\bf fqegp:} Specifies the gas phase energy, in Kelvin, of the
fluctuating charge model. This serves as a required correction term in the
calculation of the fluctuating charge Coulombic energy. 
Must be {\bf nmolty} values.

\noindent{\bf nchoiq:} This variable specifies the number of
charge updates to attempt during an ANES-MC move. Default: 1. 
Must be {\bf nbox} values. 
Note: 
If {\bf fqtemp} $>$ 0, this value should be optimized for a particular system
(5 or 10 are good initial guesses), but if {\bf fqtemp} $<$ 0, this value 
should be 1.

\noindent{\bf nswapq:} This variable specifies the number of
charge updates to attempt after a swap or swatch move using ANES-MC. 
Default: 1. Note: 
If {\bf fqtemp} $>$ 0, this value should be optimized for a particular system
(500 is a good initial guess), but if {\bf fqtemp} $<$ 0, this value 
should be 1.

\subsubsection{Section \textbf{MC\_FLUCQ}}

This section is only required if {\bf lflucq} is TRUE for at least one molecule type.
The variables included here are additional parameters needed for the
FQ potential calculation. There are two parts to the section: variables specific 
to each bead and variables specific to each bead-bead interaction.
Because of the way the loops are performed during the energy calculation, there 
must be one line for every bead (every {\bf ntii}) and every bead-bead interaction 
(every {\bf ntij}) in the system, regardless of whether or not a fluctuating charge
is involved.

\noindent {\bf Example \#1}\hfill\break
TIP4P-FQ in a pure system (1 box NVT):

\begin{verbatim}
MC_FLUCQ
! fqmolty  fqtyp  xiq(ntii)  jayself(ntii)
     1      1    0.0000       0.0000
     1      2    0.0000       88814.8
     1      3    34464.0      93494.56
! fqmolty  fqcrosstyp  jayq(ntij)
     1      1    0.0000
     1      2    0.0000
     1      3    0.0000
     1      4    0.0000
     1      5    102451.52
     1      6    144116.48
     1      7    0.0000
     1      8    144116.48
     1      9    0.0000
END MC_FLUCQ
\end{verbatim}

\noindent Note that only the H-M (ntij=6), M-H (ntij=8) and H-H (ntij=5) cross-bead 
interactions have meaning in this example.

\noindent {\bf Example \#2}\hfill\break
TIP4P-FQ + Helium vapor in an interfacial system (2 box NpT-Gibbs):

\begin{verbatim}
MC_FLUCQ
! fqmolty  fqtyp  xiq(ntii)  jayself(ntii)
     1      1    0.0000       0.0000  # O
     1      2    0.0000       88814.8  # H 
     1      3    34464.0      93494.56  # M
     2      4    0.0000       0.0000  # He
! fqmolty  fqcrosstyp  jayq(ntij)
     1      1    0.0000  # O-O
     1      2    0.0000  # O-H
     1      3    0.0000  # O-M
     2      4    0.0000  # O-He
     1      5    0.0000  # H-O
     1      6    102451.52  # H-H
     1      7    144116.48  # H-M
     2      8    0.0000  # H-He
     1      9    0.0000  # M-O
     1     10    144116.48  # M-H
     1     11    0.0000  # M-M
     2     12    0.0000  # M-He
     2     13    0.0000  # He-O
     2     14    0.0000  # He-H
     2     15    0.0000  # He-M
     2     16    0.0000  # He-He
END MC_FLUCQ
\end{verbatim}

\noindent{\bf fqmolty:} Molecule type number. A temporary variable only used to 
make the input more readable.

\noindent{\bf fqtyp:} Line number where each bead occurs in the ATOMS section 
of {\bf topmon.inp}. Equivalent to {\bf ntii}.
A temporary variable only used to make the input more readable.

\noindent{\bf xiq:} Electronegativity parameter for each bead in a fluctuating charge
model. Must be one value for each bead type in the system.

\noindent{\bf jayself:} Hardness parameter for each bead in a fluctuating charge
model. Must be one value for each bead type in the system.

\noindent{\bf fqcrosstyp:} Bead-bead cross type. Equivalent to {\bf ntij} where
ntij = (ntii - 1)*nntype + ntjj. Must be the same value as would be calculated by the
{\tt type\_2body(ntii,ntjj)} function. 

\noindent{\bf jayq:} Intramolecular Coulomb interaction parameter for each 
bead-bead cross interaction in a system with fluctuating charges. 
Must be one value for each {\bf ntij} value in the system.

\subsubsection{Namelist \textbf{mc\_gcmc}}
\noindent{\bf nequil:} Number of cycles considered
equilibration for GCMC histogram reweighting analysis, after
which data are accumulated.

\noindent{\bf ninstf:} Frequency for writing out {\bf
  nfl\#\#.dat}, which includes the step number, numbers of
molecules of each type, and total energy.

\noindent{\bf ninsth:} Frequency for accumulating the
histograms of number of molecules of each type in the
system.

\noindent{\bf ndumph:} Frequency for writing out {\bf
  his\#\#.dat}, which includes molecule numbers and the
corresponding energy, and {\bf n\#dis\#\#.dat} (one file per
molecule type), which includes the histogram of the number
of molecules of a particular type.

\noindent{\bf B:} Chemical potentials, as given by
$\exp(\beta\mu_i)=\Lambda_i^3\rho_i/\int\,\mathrm{d}\mathbf{s}_i\exp[-\beta U^{\text{intra}}(\mathbf{s}_i)]$.
The contribution due to intramolecular interactions can be
determined by simulating an isolated chain without tail
corrections and Ewald summations. The entire expression can
also be determined conveniently from a GCMC or GEMC
simulation of a single chain by repeatly trying to insert it
into box 1 without ever accepting the move, which can be
achieved by setting either the chemical potential (GCMC) or
the biasing potential (GEMC) to very unfavorable values.
Suppose you want to use GCMC to calculate the amount of
adsorption at a given pressure, $p_i$, first convert $p_i$
to fugacity, $f_i$, which would be the pressure if the
system was an ideal gas, then calculate the box length that
corresponds to a single molecule at that fugacity using the
ideal gas law: $L=(k_{\text{B}}T/f_i)^{1/3}$, and finally
perform a GCMC or GEMC simulation as mentioned above using
$L$ as the box length for box 1 and at temperature $T$. The
chemical potential from the output can be directly used as
the input for the subsequent GCMC simulations of adsorption.

\subsubsection{Namelist \textbf{mc\_ee}}
\noindent{\bf pmexpc:} Specifies the probability to perform
an expanded ensemble move that changes a molecule's
interaction parameters.

\noindent{\bf pmeemt:} Once an expanded ensemble move has
been selected, {\bf pmeemt} specifies the probability to
carry it out on a particular molecule type. Must be {\bf
  nmolty} values.

\noindent{\bf pmexpc1:} The probability for new expanded
ensemble moves. A separate input file, {\bf fort.7}, is
required.

\noindent{\bf lexpand:} If TRUE, a molecule type will be
treated with the expanded ensemble. The default setting is
FALSE. Must be {\bf nmolty} values.

\subsubsection{Namelist \textbf{mc\_simple}}
\noindent{\bf pm\_atom\_tra:} The probability to do atom
translation moves. Used to change the conformation and
center of mass of a molecule by moving individual bead(s) of
that molecule. Only for flexible molecules. Used only in
special cases. Usually 0.0d0.

\noindent{\bf Armtra:} The initial maximum translational
displacements, in~\AA, for all molecule types and all bead
types. Only used if {\bf linit} is TRUE. If {\bf linit} is
FALSE, it is read from {\bf file\_restart}.

\noindent The target acceptance ratio for atom translation
moves is always 0.5.

\noindent{\bf pmtra:} Specifies the probability to perform a
translation move that changes the position of a molecule.
Usually approximately one-third of the total moves are
translation moves.

\noindent{\bf pmtrmt:} After a translation move has been
selected, {\bf pmtrmt} specifies the probability to carry it
out on a particular molecule type. Must be {\bf nmolty}
values.

\noindent{\bf rmtra:} The initial maximum translational
displacements, in~\AA, for all molecule types. Only used if
{\bf linit} is TRUE. If {\bf linit} is FALSE, it is read
from {\bf file\_restart}.

\noindent{\bf tatra:} The target acceptance ratio for
translational moves. We used to think 0.5 is a good value, 
but now think the following:
Ilja (9/24/15):"Please take a look at 
Mountain, R. D.; Thirumalai, D. Physica A: Statistical Mechanics and its Applications
1994, 210, 453–460.
Although MCCCS-MN is much more complex, I think that our
recommended target acceptance rate for translation, rotation,
and volume moves should be about 30\%.  [50\% is almost certainly too high.]"

Note that, although not specified, the fraction of rotation
moves is equal to $1 - {\bf pmtra}$.

\noindent{\bf pmromt:} After a rotation move has been
selected, {\bf pmromt} specifies the probability to carry it
out on a particular molecule type. There must be {\bf
  nmolty} values.

\noindent{\bf rmrot:} The initial maximum rotational
displacements, in radians, for all molecule types. Only used
if {\bf linit} is TRUE. If {\bf linit} is FALSE, it is read
from {\bf file\_restart}.

\noindent{\bf tarot:} The target acceptance ratio for
rotational moves. We used to think 0.5 is a good value,
but now think the following:
Ilja (9/24/15):"Please take a look at 
Mountain, R. D.; Thirumalai, D. Physica A: Statistical Mechanics and its Applications
1994, 210, 453–460.
Although MCCCS-MN is much more complex, I think that our
recommended target acceptance rate for translation, rotation,
and volume moves should be about 30\%.  [50\% is almost certainly too high.]"

If you have specific atom translations that you want to use you can set up the 
section as follows:
\begin{verbatim}
SPECIFIC_ATOM_TRANSL
! How many atoms should we be performing translations on?
3
! What is the index of those atoms in their respective molecules?
1 2 3
! What is the molecule type for each atom?
1 1 1
END SPECIFIC_ATOM_TRANSL
\end{verbatim}

In this example we will be performing this move type on 3 different beads as indictaed by 3 in
the first noncomment input line after the section header. The next input line specifies the index
of those beads within their molecule. In this example we only move the first 3 beads in a molecule. 
The last line specifies which molecule the bead belongs to. It is convienient to keep the molecule
number and the bead number vertically alligned so you can tell which molecules and which beads are
having this move type applied.  


\subsection{Section \textbf{SIMULATION\_BOX}}
For each of the \textbf{nbox} simulation boxes, including
the following:

\noindent{\bf boxlx, boxly, boxlz:} Specifies the $x, y$ and
$z$ dimensions, in~\AA, of the simulation box. Used only for
initialization. Note that the box length must be greater
than twice the cutoff ({\bf rcut}). If, for example, {\bf
  rcut} is 14~\AA, it is a good idea to start with a box
length of slightly more than 2{\bf rcut} (for example,
32~\AA) to ensure no volume moves would reduce the box
length to less than 28~\AA. If such a move is attempted, the
simulation will either abort or continue depending on the
setting {\bf allow\_cutoff\_failure:}.

\noindent{\bf rcut:} This is the cutoff for the pairwise van
der Waals and Coulomb interactions. Interactions beyond {\bf
  rcut} are not calculated (explicitly). A smaller value
might be used for equilibration than for production. The
TraPPE force field uses a 14~\AA-cutoff. In the simulation
of a large vapor box with the Ewald sum, if {\bf
  L\_Ewald\_Auto} is set to TRUE, it is necessary to set
{\bf rcut} to be 35-40\% of {\bf boxlx}. This reduces {\bf
  kalp}, which in turn decreases the number of vectors
required (upper limit {\bf vectormax} defined in module {\bf
  energy\_kspace}). Alternatively, you can set {\bf
  L\_Ewald\_Auto} to FALSE, and use a small {\bf kalp} (just
for this vapor box)

\noindent{\bf kalp:} This is the Ewald sum convergence
parameter. It will be calculated automatically as 3.2
divided by {\bf rcut} if {\bf L\_Ewald\_Auto} is TRUE. If
{\bf L\_Ewald\_Auto} is FALSE, the value of the Ewald sum
convergence parameters will be taken from this variable. If
all charge interactions are being calculated (\textit{i.e.},
{\bf lchgall} is TRUE), then {\bf kalp} should be greater
than or equal to 5.6. If a charge cutoff is being used
(\textit{i.e.}, if {\bf lchgall} is FALSE), then {\bf kalp}
should be greater than or equal to 3.2 divided by {\bf
  rcut}. Verification of the convergence of the Ewald sum is
sometimes needed.

\noindent{\bf rcutnn:} The nearest neighbor bead-bead cutoff
for the neighbor list. Not normally used.

\noindent{\bf numberDimensionIsIsotropic:} The number of
dimensions that are isotropic. Its value can be 3 (
indicating the system is isotropic in all directions), 2
(isotropic in the x and y directions), or any other number
(non-isotropic). It is customary to take the z axis as the
special direction in a system that is isotropic in two
directions (e.g. slit pore, external field). When this
parameter is set to 2 or 3, the maximum displacements for
atom translation, COM translation, and rigid-body rotation
will be averaged across all the equivalent directions during
their updating and when reading the restart file. This value
must be set appropriately for production runs, but in
initial equilibration one may use different maximum
displacements to allow for rapid equilibration and status
monitoring.

\noindent{\bf lsolid:} If TRUE, the $x, y$ and $z$ box
lengths can vary independently. Only to be used if the box
can withstand anisotropic stress (\textit{e.g.}, for a solid
phase.

\noindent{\bf lrect:} Only functional if {\bf lsolid} is
also TRUE, in which case the simulation box will remain
orthorhombic (\textit{i.e.}, the axes will remain mutually
perpendicular). This variable has no effect if {\bf lsolid}
is FALSE. Note that in the case of a non-orthorhombic
system, {\bf linit} must be FALSE and the system must start
from {\bf file\_restart}. In the {\bf file\_restart} one must
specify maximum displacements for the H-matrix elements
(9 values) after maximum volume displacements. This means
non-orthorombic box displacemnent will become a dummy variable.
Later, H-matrix has to be specified after maximum displacements.

\noindent{\bf lideal:} If TRUE, the box is treated as an
ideal gas, and pair interactions are not computed. In
addition, tail corrections are not applied (this is
accounted for also in the {\bf swap} and {\bf swatch}
subroutines).

\noindent{\bf ltwice:} If TRUE, the minimum image convention
is applied twice in this box. Necessary for the smaller
(4x6; 20x26~\AA$^2$) RPLC set-up.

\noindent{\bf temp:} The temperature in K.

\noindent{\bf express:} The external pressure, in MPa, for
each box in an $NpT$ or $NpT$-Gibbs ensemble simulation.

\noindent{\bf ininch:} The inital number of molecules of each
type in this box. There must be {\bf nmolty} values.

Note that {\bf moltyp} (number of molecules of each type)
has been retired. They are now simply calculated as sum of
ininch of each box.

\noindent{\bf ghost\_particles:} The number of ideal gas
particles for this box. Usually set to zero. The ideal gas
particles do not interact with the rest of the system;
rather, they contribute only to the ideal gas part of volume
entropy and pressure. Ideal gas particles can be used to
maintain the specified external pressure in a $NpT$(-Gibbs)
ensemble simulation when the sum of the vapor pressure of
all the components in the system is smaller than the
external pressure. The effect can also be accomplished by
using helium atoms in the vapor phase, which increases the
computational cost but also takes up real volume.

\noindent{\bf inix, iniy, iniz:} Integer numbers of how many
molecules should be placed along the $x, y$ and $z$
directions, respectively, in the initial lattice. If the box
is cubic, these three values should be relatively close to
one another.

\noindent{\bf inirot:} This value specifies the initial
rotation along the $z$ axis of each new atom. This applies
only to linear molecules and influences the torsional angle.
If {\bf inirot} is 0 then each atom is randomly rotated. If
{\bf inirot} is greater than 0, each atom is uniformly
rotated by {\bf inirot} degrees. If {\bf inirot} is less
than 0, the atoms are alternatively rotated in the $z$
direction by $\pm$\textbf{inirot}.

\noindent{\bf inimix:} For mixture simulations, this
variable specifies the order in which to select molecules
for the initial lattice. If {\bf inimix} is 0, molecules are
placed randomly. If {\bf inimix} is greater than 0,
molecules are placed in order by type. If {\bf inimix} is
less than 0, molecules are placed in alternating order by
type.

\noindent{\bf zshift:} This value specifies a uniform shift
in the $z$ direction so that terminal beads have a certain
$z$ coordinate. Used only for monolayer calculations. {\bf
  zshift} should be 0 otherwise.

\noindent{\bf dshift:} This value specifies the amount, in
\AA, to stagger each row in the $x$ direction of the initial
lattice. Optimal spacing can be achieved by setting this
value to half of the spacing between molecules, which can be
calculated dividing {\bf boxlx} by {\bf inix}.

Note: \textbf{inix} through \textbf{dshift} are used only
for initialization (\textit{i.e.}, when {\bf linit} is
TRUE). Even if {\bf linit} is FALSE, however, these
variables must still be included.

\noindent{\bf use\_linkcell:} If TRUE, a linked cell
structured will be used to increase efficiency for this box.
This variable should be used only for very large systems and
only if {\bf rcut} is less than $0.25~{\bf boxlx}$.

\noindent{\bf rintramax:} Specifies the maximum distance
between the ends of a molecule for use in the link cell
list. One value only (for the largest molecule in the
system).

Currently only one box can be treated with cell structure.

\subsection{Section \textbf{MOLECULE\_TYPE}}
Each type of molecule must have its own specification
section in which the connectivity and potential parameters
are specified. Note that a ``bead'' is one unit of a
molecule. It may be a single atom or a group of atoms. For
example, in a united atom model, a $CH_x$ group is
considered to be a single bead.

Certain types of molecules should be numbered in particular
ways to increase efficiency. For branched molecules, it is
often best to start numbering at the most highly branched
site. For example, for methyl acrylate, the carbonyl carbon
should be bead number 1 and the beads to which it is bonded
should be numbers 2, 3 and 4. For alcohol molecules and
water, it is usually best to number the polar oxygen as bead
number 1 so its location is determined first in a swap move.

The following is repeated {\bf nmolty} times:

\subsubsection{Molecule Parameters}

\noindent{\bf molecname:} Name of the molecule. Max length
is 10, and must contain at least one English letter. This
variable is optional, if it is not specified then the molecule
will have "undefined" as a name.

\noindent{\bf nunit:} The number of beads in the molecule.

\noindent{\bf nugrow:} The maximum number of beads to regrow
with CBMC. Should be equal to {\bf nunit} unless {\bf
  explct.f} is used.

\noindent{\bf ncarbon:} For explicit hydrogen models, the
number of carbon atoms present. This allows for a simplified
setup of the exclusion table. Note that these atoms should
be listed first in the molecule specification section of
{\bf file\_input}. {\bf ncarbon} should be equal to {\bf
  nunit} for all other models.

\noindent{\bf maxcbmc:} The maximum number of beads that are
allowed to grow with CBMC. Should be equal to {\bf nugrow}
if the whole molecule needs to be grown.

\noindent{\bf maxgrow:} The maximum number of interior
segments to consider in a SAFE-CBMC regrowth (see {\bf
  pmfix}).

\noindent{\bf iring:} The number of ring or interior atoms 
that needs to be regrown with SAFE-CBMC.

\noindent{\bf lelect:} set to TRUE if the molecule contains
one or more beads with partial charges.

\noindent{\bf lring:} Specifies whether the molecule is a
flexible ring that should be regrown with SAFE-CBMC during a
swap move.

\noindent{\bf lrigid:} If TRUE, part or all of this molecule
should be treated as a rigid body. Note that if {\bf lrigid}
is TRUE, {\bf growpoints} must be included. If {\bf lrigid}
is FALSE, {\bf growpoints} must not be included. Typically
{\bf lrigid} and {\bf lbranch} are both TRUE or both FALSE.

\noindent{\bf lbranch:} If TRUE, the structure of a molecule
must be read from {\bf file\_struct}. If FALSE, the
structure can be grown with CBMC and {\bf file\_struct} is
not required. This variable is usually FALSE unless CBMC
cannot grow a particular molecule type. Typically {\bf
  lrigid} and {\bf lbranch} are both TRUE or both FALSE.

\noindent{\bf lrplc:} NOTE: This variable is no longer used! See {\bf lhplc}.

%% \noindent{\bf hplc:} Specifies whether the simulation
%% contains HPLC bonded chains.

%% \noindent{\bf lmovie\_vrb:} Controls verboseness of movie
%% file. If FALSE, doesn't print connectivity.

%% \noindent{\bf lhplc\_c:} If {\bf lhplc} is TRUE, this must
%% be included. If TRUE, the molecule type is an HPLC bonded
%% chain and special rules are applied for SAFE-CBMC.

%% \noindent{\bf nhplc\_sca, hplc\_sca:} If {\bf lhplc\_c}
%% is TRUE, these must be included. They are specified
%% together on a new line between the {\bf maxgrow} and {\bf
%% lq14scale} lines. {\bf nhplc\_sca} is the number of side
%% chain atoms in the molecule and {\bf hplc\_sca} are their
%% bead numbers. For example, if the bonded chain were
%% O$_{3}$-Si-O-Si[(CH$_{3}$)$_{2}$]-C8, {\bf nhplc\_sca}
%% would be two, and the bead numbers for the two bracketed
%% CH$_{3}$ groups would be provided for {\bf hplc\_sca}.

\noindent{\bf lsetup:} If TRUE, try to automatically setup
the bead-bead potential types. Only for linear molecules.
Experimental. Use with caution.

\noindent{\bf lq14scale:} If TRUE, 1-4 charge-charge
interactions need to calculated. Usually TRUE for the TraPPE
force field. TRUE for acrylates and diols.

\noindent{\bf qscale:} If {\bf lq14scale} is TRUE, {\bf
  qscale} gives the scaling factor for the 1-4 charge-charge
interactions. Generally 0.5.

\noindent{\bf iurot:} For a single rotation center, {\bf
  iurot} specifies the bead around which to perform
rotations. 0 specifies the center of mass and should be used
as the default. For multiple rotation centers, set {\bf
  iurot} $< 0$ and add values for \textbf{nrotbd},
\textbf{irotbd} and \textbf{pmrotbd} immediately after all
of the variables in section \ref{avbmc_vars} (immediately
after {\bf pmbias2}). If {\bf iurot} $\ge$ 0, then these
three variables are not included.

\noindent{\bf isolute:} Specifies how often to write the
configuration of this molecule to the {\bf file\_solute} for
later analysis. Similar to {\bf imv} but specific to this
type of molecule. If not needed, a value larger than {\bf
  ncycles} can be used.

\noindent{\bf eta:} Specifies a uniform background potential
(in Kelvin) added to this molecule in each simulation box.
Usually 0. There must be one value for each simulation box.

\noindent{\bf growpoints:} Specifies the number of sites in
a partially rigid molecule from which CBMC growths should be
attempted. Only needed if {\bf lrigid} is set to TRUE. Note
that an error will occur if {\bf growpoints} is included
while {\bf lrigid} is FALSE. If non-zero, {\bf growpoints}
must be followed by the unit numbers of the growth sites
with one site listed per line. In a partially rigid
molecule, the flexible part must be numbered first. For
example, only the nitro group in nitrotoluene is flexible.
The two oxygen beads are given numbers 1 and 2 and the
nitrogen atom is assigned bead number 3. There is one
growpoint at bead number 3.

\subsubsection{Bead Specification and Connectivity}
\label{beadspec}
The variables in this section must be included for each bead
in the molecule.

\noindent{\bf unit:} The bead number.

\noindent{\bf ntype:} bead type as specified in \ref{atoms}.

\noindent{\bf leaderq:} specifies the unit number of the
first bead in a neutral charge group which {\bf unit}
belongs to. It is used only if {\bf lchgall} and {\bf
  lewald} is FALSE, in which case it should be set to the
lowest unit number in the neutral charge group; otherwise
set to {\bf unit}.

\noindent{\bf stretching:} This section specifies the bonds
formed with this bead. The first line is the number of bonds
and is followed by one line for each bond that contains two
values: first the unit number to which {\bf unit} is bonded
and the bond type as given in \ref{bonds}.

\noindent{\bf bending:} This section specifies the bond
bending interactions that have this unit as the first or the
last bead. The organization of this section is similar to
the structure of {\bf stretching}: the first line specifies
the number of bond bending interactions and the following
lines specify the two other units involved in the bending
(the unit closest to {\bf unit} should be listed first)
followed by the bend type as given in \ref{angles}.

\noindent{\bf torsion:} This section specifies the torsional
interactions that have this unit as the first or the last
bead. The organization of this section is similar to the
structure of {\bf stretching} and {\bf bending}: the first
line specifies the number of torsional interactions and the
following lines specify the three other units involved in
the torsion in the order of decreasing proximity to this
bead followed by the torsion type as given in
\ref{dihedrals}. Note that if a torsion is defined,
stretching and bending interactions involing the constituent
four beads must also be included, even if their force
constants are zero.

\subsubsection{Additional {\bf iurot} Information}
If {\bf iurot} is less than 0, the following three variables
must follow \ref{beadspec}. If {\bf iurot} is greater than
or equal to 0, these variables must not be included.

\noindent{\bf nrotbd:} The number of rotation centers.

\noindent{\bf irotbd:} The unit numbers of the rotation
centers. 0 denotes the center of mass. There must be {\bf
  nrotbd} values.

\noindent{\bf pmrotbd:} The probability to rotate around
each center. There must be {\bf nrotbd} values.

\subsection{Section \textbf{INTERMOLECULAR\_EXCLUSION}}
Special rules for exclusion of intermolecular interactions.
Each line specifies one rule: the molecule type and unit and
the molecule type and unit interactions between which should
be excluded.

\subsection{Section \textbf{INTRAMOLECULAR\_SPECIAL}}
\label{intra_special}
By default, intramolecular van der Waals interactions are
considered only if beads are seperated by four or more bonds
and intramolecular Coulomb interactions are considered for
beads separated by three (1-4) or more bonds with 1-4
interactions scaled by {\bf q14scale}. This section
specifies exceptions to this convention.

Each line gives the molecule type, the two bead numbers to
change the rule for, +1 or -1 (for turning the interaction
on or off, respectively), the factor to scale the L-J
interaction by and the factor by which to scale the
coulombic interaction.

\subsection{Section \textbf{INTRAMOLECULAR\_OH15}}
Specifies whether an additional repulsive potential should
be added for intramolecular hydroxyl hydrogen--oxygen pairs
separated by 4 bonds (1-5).

Each line lists the molecule type, the two bead numbers and
the a15 value, which is 1 is for an ether oxygen ($4\times
10^7$) or 2 is for an alcohol oxygen ($7.5\times 10^7$).

%%%%%%%%%%%%%%%%%%%%%%%%%%%%%%%%%%%%%%%%%%%%%%%%%%%%%%%
\section{Other Input and Output Files}

\subsection{Input Files}

\noindent{\bf file\_restart:} This is the restart file and
must be present for all simulations in which {\bf linit} (in
{\bf file\_input} is FALSE. To continue a simulation, copy
the final configuration file from the previous run, {\bf
  config\#\#.dat}, to {\bf file\_restart}. The following
list includes, in order, the information that is contained
in {\bf file\_restart}. Multiple values listed on a single
line (for example, the maximum rotational values in the $x$,
$y$ and $z$ directions) are separated by spaces, not commas
or any other characters.
\begin{center}
\begin{tabular}{| l |}
\hline
total number of cycles for which the simulation has run \\ \hline
max atomic displacements in the $x$, $y$ and $z$ directions  \\ \hline
max translational values in the $x$, $y$ and $z$ directions for molecules of type 1 in box 1 \\ \hline
max rotational values in the $x$, $y$ and $z$ directions for molecules of type 1 in box 1 \\ \hline
max translational values in the $x$, $y$ and $z$ directions for molecules of type 2 in box 1 \\ \hline
max rotational values in the $x$, $y$ and $z$ directions for molecules of type 2 in box 1 \\ \hline
  . . . \\ \hline
max translation values in the $x$, $y$ and $z$ directions for molecules of type {\bf nmolty} in box 1 \\ \hline
max rotation values in the $x$, $y$ and $z$ directions for molecules of type {\bf nmolty} in box 1 \\ \hline
  . . . \\ \hline
max translation values in the $x$, $y$ and $z$ directions for molecules of type {\bf nmolty} in box {\bf nbox} \\ \hline
max rotation values in the $x$, $y$ and $z$ directions for molecules of type {\bf nmolty} in box {\bf nbox} \\ \hline
fluctuating charges for molecule types 1 through {\bf nmolty} in box 1 \\ \hline
  . . . \\ \hline
fluctuating charges for molecule types 1 through {\bf nmolty} in box {\bf nbox} \\ \hline
max volume displacements for boxes 1 through {\bf nbox} \\ \hline
$x$, $y$ and $z$ lengths of box 1  \\ \hline
  . . . \\ \hline
$x$, $y$ and $z$ lengths of box {\bf nbox} \\ \hline
total number of molecules in the simulation ({\bf nchain}) \\ \hline
number of types of molecules ({\bf nmolty}) \\ \hline
number of units for each molecule type ({\bf nmolty} values) \\ \hline
molecule type of each chain ({\bf nchain} values) \\ \hline
box number of each chain ({\bf nchain} values) \\ \hline
$x$, $y$ and $z$ coordinates and the charge, $q$, of molecule 1 \\ \hline
$x$, $y$ and $z$ coordinates and the charge, $q$, of molecule 2 \\ \hline
  . . . \\ \hline
$x$, $y$ and $z$ coordinates and the charge, $q$, of molecule {\bf nchain} \\
  \hline
\end{tabular}
\end{center}

\noindent{\bf file\_struct:} Only necessary if {\bf lbranch}
in {\bf file\_input} is TRUE for any molecule(s), in which
case the structure(s) need be specified here. This is done
by 1 comment line, followed by lines for each bead consisting 
of by xyz-coordinates of each bead. For TIP4P, ordering the 
beads O, H, H, M, for example, 
\begin{verbatim}
this line needed for comments
 0.0 0.0 0.0
 0.9572 0.0 0.0
 -0.2400 0.926627 0.0
 0.0918118904 0.118619462 0.0
\end{verbatim}
One example in the test-suite, TEST6, has a 4th column, but
this column is apparently not read.


\noindent{\bf fort.7, fort.44:} Used for the new and the old
expanded ensemble calculations, respectively. {\bf fort.7}
holds the current values of the Lennard-Jones parameters.

\noindent{\bf fort.23:} Only necessary if SAFE-CBMC is used.
This file contains the probability histograms used in that
algorithm. To adapt the probabilities, the final histograms
that are given in the {\bf fort.21} output file should be
copied to {\bf fort.23} before continuing the simulation.

\noindent{\bf fort.40, fort.41, fort.42:} Tabulated potential
files for intramolecular bonded interactions all have the
same format (make sure potential does not go up to
infinity!):
\begin{verbatim}
number_of_tabulated_potentials
potential_number
r1 U_1 (separation energy)
r2 U_2
...
rn U_n
1000 1000 (termination marker)
(repeat the above for each additional potential)
\end{verbatim}

\noindent{\bf fort.43, fort.44:} Tabulated potential files
for van der Waals and Coulomb interactions, respectively.
Both files have the following format:
\begin{verbatim}
number_of_tabulated_potentials
beadtype_1 beadtype_2 (for unlike interactions, list both 1-2 and 2-1)
r1 Uq_1 (separation energy)
r2 Uq_2
...
rn Uq_n
1000 1000 (termination marker)
(repeat the above for each additional potential)
\end{verbatim}

\noindent{\bf fort.61, fort.62, fort.63:} The vibration,
bending and torsion libraries for the molecule builder.
Experimental - use with caution.

\noindent{\bf zeolite.cssr, zeolite.pdb, zeolite.cif:} Used
in zeolite/MOF calculations to specify the structure of the
rigid frameworks. Please see \ref{zeolite} for more
information.

\noindent{\bf zeolite.ztb:} Used in zeolite/MOF calculations
to store the tabulated potential energy. Please see
\ref{zeolite} for more information.

\subsection{Output Files}
\label{output}
Note that the two characters included in the names of many
of the output files (denoted here as \#\#) are the {\bf
  run\_num} and {\bf suff} from {\bf file\_input}.

\noindent{\bf run\#\#.dat:} The majority of the information
about the simulation including the fraction of accepted
moves and the calculated energies and densities is written
to {\bf run\#\#.dat}.

\noindent{\bf config\#\#.dat:} The final configuration of the
system as well as the maximum displacements and box lengths
are contained in this file, which is also known as the
restart file. Its format is identical to the format of {\bf
  file\_restart}. To continue a simulation, this file should
be copied to {\bf file\_restart}.

\noindent{\bf movie\#\#.dat:} The movie file which holds the
configurations of the system as output every {\bf imv}
cycles. Used to determine the radial distribution functions,
$g(r)$ among other things. Formerly called {\bf fort.10}.

\noindent{\bf file\_solute:} The solute movie file output
every {\bf isolute} cycles. Output more frequently to obtain
information about a particular molecule in the simulation
with better statistics ({\it e.g.}, the solute molecules in
a chromatography simulation).

\noindent{\bf file\_traj:} Contains the box lengths, total
energy, pressure, and number of each molecule type for each
box, output every cycle. Useful for many things.

\noindent{\bf cell\_param\#\#.dat:} If there is a
non-orthorhombic box in the simulation, this file will
contain the cell lengths and the angles between cell
vectors, written out every cycle.

\noindent{\bf box\#config\#\#.xyz:} Lists the xyz coordinates
and identifies the molecule type of each molecule in the
system. In a format for VMD to visualize.

\noindent{\bf fort.14, fort.15, fort.16, fort.17, fort.18:}
If the simulation is of a single molecule type and {\bf
  idiele} is less than {\bf nstep}, then these 5 files will
contain averages related to the dielectric constant. {\bf
  fort.14} and {\bf fort.15} will have
\begin{equation}
\frac{4\pi}{3Vk_{\rm B}T} \frac{1}{4\pi\epsilon_0}
\langle {\bf M}^2\rangle
\text{  and  }
\frac{4\pi}{3Vk_{\rm B}T}\frac{1}{4\pi\epsilon_0}
\left(\langle {\bf M}^2\rangle -
\langle {\bf M}_x\rangle^2 -
\langle{\bf M}_y\rangle^2 -
\langle{\bf M}_z\rangle^2 \right)
\end{equation}
respectively, where {\bf M} is the total dipole moment of
the box and {\bf M}$_x$, {\bf M}$_y$, and {\bf M}$_z$ its
$x, y$ and $z$ components. Both quantities are unitless.
Note that Allen \& Tildesley relate the dielectric constant
of the system $\epsilon$, to these quantities {\sl plus} 1.
{\bf fort.16, fort.17}, and {\bf fort.18} will contain
$\langle{\bf M}_x\rangle, \langle{\bf M}_y\rangle,$ and
$\langle{\bf M}_z\rangle$, respectively, in units of
$e$~\AA.

\noindent{\bf fort.14, fort.15, fort.16, fort.17, fort.18, fort.19:}
If the simulation is of a single molecule type, {\bf idiele} is less
than {\bf nstep} and {\bf lnpt} is TRUE, then these 6 files will
contain the following averages: $\langle V\rangle, \langle V^2
\rangle, \langle U \rangle, \langle U^2\rangle, \langle V
\rangle\langle U \rangle,$ and $\langle VU \rangle - \langle V \rangle
\langle U \rangle$ where $V$ is the box volume and $U$ is the total
energy, for each box.  This is in addition to the dipole moment
information contained in {\bf fort.14}-{\bf 18} (see above).

\noindent{\bf fort.21:} The final SAFE-CBMC probability
histograms are contained in this file, if the SAFE-CBMC
algorithm is used. Can be used to start a simulation with
the new histograms by copying to {\bf fort.23}.

\noindent{\bf fort.27:} For dielectric constants. For an
$NVT$ simulation with ldielect=TRUE, this file contains
the $x$, $y$, and $z$ components of the system dipole moment
in units of $e\cdot$\AA. The quantities may be used to
calculate the dielectric constant according to
\begin{equation}
\epsilon = 1 + \frac{1}{4\pi\epsilon_0} \frac{4 \pi}{3VT} \left ( \langle \mu^2 \rangle - \langle \mu_x^2 \rangle
 \langle \mu_y^2 \rangle - \langle \mu_z^2 \rangle \right )
\end{equation}
(See Allen and Tildesley, page 161.)

\noindent{\bf fort.31, fort.32, fort.33:} If the AVBMC
algorithm is used, these files will contain the cluster
statistics and energy ratios, which are output every cycle.

\noindent{\bf fort.55, fort.56:} If {\bf iheatcapacity} is
less than {\bf nstep} and {\bf lnpt} is FALSE, then {\bf
  fort.55} contains the following averages: $<E>$ and
$<E^2>$. If {\bf iheatcapacity} in {\bf file\_input} is less
than {\bf nstep} and {\bf lnpt}, which is in {\bf
  control.inc}, is TRUE, then {\bf fort.56} contains the
following averages: $<H>$ and $<H^2>$.

\noindent{\bf end2end\_box1, end2end\_box2, end2end\_box3:}
These files contain the end to end vector distribution for
box 1, box 2 and box 3 respectively. If the box is not
present in simulation then that particular file will be
empty.

\noindent{\bf rhoz\_box1, rhoz\_box2, rhoz\_box3:} These
files contain the $z$ density profiles for the respective
boxes.

\noindent{\bf comrhoz\_box1, comrhoz\_box2, comrhoz\_box3:}
These files contain center of mass z density profile for the
respective boxes.

\noindent{\bf beadrdf\_box1, beadrdf\_box2, beadrdf\_box3:}
These files contain bead-bead readial distribution functions
for the the respective boxes.

\noindent{\bf comrdf\_box1, comrdf\_box2, comrdf\_box3:}
These files contain center of mass readial distribution
functions for the the respective boxes.

\noindent{\bf beadnum\_box1, beadnum\_box2, beadnum\_box3:}
These files contain bead-bead number integrals for the the
respective boxes.

\noindent{\bf comnum\_box1, comnum\_box2, comnum\_box3:}
These files contain center of mass number integrals for the
the respective boxes.

\noindent{\bf bendang\_dist\_box1, bendang\_dist\_box2,
  bendang\_dist\_box3:} These files contain bending angle
distribution for the respective boxes.

\noindent{\bf tors\_frac\_box1, tors\_frac\_box2,
  tors\_frac\_box3:} These files contain torsion fractions
for the respective boxes.

\noindent{\bf torsprob\_box1, torsprob\_box2,
  torsprob\_box3:} These files contain torsion angle
probabilty distribution vs number of defects per chain, for
the respective boxes.

\noindent{\bf Gauchedefects\_box1, Gauchedefects\_box2,
  Gauchedefects\_box3:} These files contain fraction gauche
defects versus torsion for the respective boxes.

\noindent{\bf pattern\_box1, pattern\_box2, pattern\_box3:}
These files contain pattern of g+, trans, g- for the
respective boxes. Transform first number into base 3 to get
the pattern of gauche defects where -1=g+, 0 = trans, 1=g+.

\noindent{\bf decoder:} This file contains decoder
information for the gauche defect pattern.

\subsection{Notes on File Management}
\noindent Here are some basic guidelines about file
management. To prevent output files from being overwritten,
change the {\bf run\_num} and/or the {\bf suff} in {\bf
  file\_input}. Note that output files whose titles do not
contain {\bf run\_num} and {\bf suff} will be overwritten
during each new run. Rename them to avoid this.

\subsubsection{Force Field Development} If one is developing a force
field for a particular molecule then it is important to keep
{\bf file\_run}. It is not necessary to keep all the other
output files. If you believe that you have parameters that
are close to the desired value then you should carry out
each simulation in a separate directory say T1, T2 (trial 1,
trial 2), etc., and keep all the files for future reference.
It is also a good idea to have a README file that has some
basic iformation as to what is there in that particular
directory.

\subsubsection{Application-Oriented Simulations} If the objective of
the simulation is to to provide microscopic understanding of
the system then you should {\bf save all the files} of your
simulation. You might want to create new directory for
different simulation runs and copy the old files to files
with names appended by date or any other way you want to
identify them.

%%%%%%%%%%%%%%%%%%%%%%%%%%%%%%%%%%%%%%%%%%%%%%%%%%%%%%%
\section{Example Simulation}
Consider the calculation of the vapor-liquid coexistence
curve for methanol. To do this, we will use the $NVT$-Gibbs
ensemble to determine the saturated liquid and vapor
densities for a range of subcritical temperatures.
Determining the vapor pressures will also allow us to
predict the normal boiling point.

\subsection{Initialization: Melting and Cooling}
First we specify the connectivity and force field parameters
in {\bf file\_input}. Since we are starting from scratch
({\it i.e.}, we do not have the {\bf file\_restart} restart
file), we set {\bf linit} to TRUE in {\bf file\_input}. In
this case, the box lengths for both simulation boxes will be
read from {\bf file\_input} and we try to pick densities
that correspond roughly to a liquid and a vapor. We place
between 10 and 33\% of the molecules in the vapor box. For
example, if we are simulating 300 molecules ({\bf nchain} =
300), then we might put 50 molecules in the vapor phase and
use 29~\AA~and 38~\AA~as box lengths for the liquid and
vapor phases, which correspond to 0.55 and
0.048~g~cm$^{-3}$, respectively.

{\bf Note:} For VLE simulations in the Gibbs ensemble, the
volume of the vapor phase should always be larger than the
volume of the liquid phase. Making the volume of the vapor
phase (highly compressible) smaller than the volume of the
liquid phase (less compressible) allows for huge
fluctuations in the volume because the volume of the liquid
has to change only by a relatively small amount to allow for
a relatively huge change in the volume of the vapor phase.
The volume of the vapor box does not need to be much
larger than the liquid box near the critical temperature.

Split up the attempted Monte Carlo move probabilities so
that we perform {\sl no} volume moves ({\bf pmvol} = 0) and
evenly divide the probability into CBMC, translation and
rotation by setting {\bf pmcb} = 0.33 and {\bf pmtra} = 0.67
and leaving all other probabilities at zero. Since we want
to ``melt'' this structure away from the initial lattice, we
run approximately 2000 cycles at a high temperature ({\it
  e.g.}, 2000 K) to create a disordered system. For this
process, a potential cutoff of 9~\AA~is sufficient.

\noindent Once this is done, we don't want to initialize the
system again, so we set {\bf linit} to FALSE and copy {\bf
  config\#\#.dat} to {\bf file\_restart}. At this point, we
want to ``cool'' the simulation, so we leave the cutoff and
probabilities the same and change the temperature to around
90\% of the critical temperature (for example, 475 K) and
then run for at least 5000 cycles. If we set {\bf iblock} to
1000, at the end of the output file we will see 5 values of
the energy (among other things) averaged over 1000 cycles
each, and we can get a feel for how the energy is changing.
It will likely be fairly steady after 2000 to 3000 cycles.

\subsection{Further Equilibration}
Now that we have brought the system to a temperature for
which we want an accurate value of the saturated densities,
we further equilibrate the system. Since we want to
equilibrate density and chemical potential, we need to add
volume and swap moves, respectively. We also want a more
accurate calculation of the potential energy, so we set {\bf
  rcut} to 14~\AA. Don't forget to copy {\bf config\#\#.dat}
to {\bf file\_restart} for each new simulation.

We want one accepted volume move per cycle at temperatures near 
the critical point, so we set {\bf pmvol} to 2 / $N_{\rm molecules}$ 
as we will be going for a target acceptance ratio of 50\% for 
volume moves ({\bf tavol} = 0.5). We set {\bf iratv} so that the maximum
volume displacement is allowed to change, for example, once
every 250 cycles. Now we want some swap moves too, so we set
{\bf pmswap} = 0.01 and run the simulation for 10000 cycles.
Like the volume move, we want one accepted every 1 cycles (at 
lower temperatures 1 accepted volume and swap move every ten cycles 
is sufficient), but we have no target acceptance ratio, so we 
have to see by trial and error what value of {\bf pmswap} is appropriate.
In general, it is best to have a similar amount of accepted swap and
volume moves.

For the stated probabilities, here's a sample of the output
file as it pertains to the swap move:

{\bf \begin{tabbing}
\#\#\# Molecule swap \#\#\# \\
\\
 molecule typ = \hskip 24pt 1 \\
between \= box 1 and 2 into box 1\\
   \> uattempts = 14021.0 attempts = 14021.0 accepted = 254.0\\
 suc.growth \% = 99.501   accepted \% = 1.812\\
between box 1 and  2 into box 2\\
   \> uattempts = 13851.0 attempts = 13851.0 accepted = 275.0\\
 suc.growth \% =100.000   accepted \% = 1.985\\
\end{tabbing} }

Because we ran 10000 cycles, we had $10000 / (254 + 275)
\approx 19 $ cycles between swap moves. Since we want that
to be closer to 1, we increase {\bf pmswap} to $0.01 \times
\frac{19}{1} = 0.19$ for the next simulation. Note that
since there is an overall flux of molecules from the liquid
to the vapor box, the simulation has not reached
equilibrium.

\subsection{Production}
Once we have stable vapor and liquid phases in our
simulation, it is time to begin production. At this point,
we should know that we will have one accepted volume move
and one accepted swap move every cycle. We set {\bf
  iratio} and {\bf iratv} to a value greater than or equal
to {\bf nstep}. If these two values change during the
production period, it would violate the principle of
microscopic reversibility. In addition, it is good practice
to set the maximum $x$, $y$ and $z$ translational
displacements in the {\bf file\_restart} file to the average
of these three values. The same should be done for the
rotational displacements.

We can start calculating the pressure more frequently now,
but it is still a fairly expensive calculation, so we set
{\bf iratp} to perhaps 5, which will calculate the pressure
every 5 cycles. The pressure of a liquid is very noisy, and
the standard deviation can be larger than the actual value!
So in a case such as this ($NVT$-Gibbs ensemble) we say that
the pressure between the two boxes is equilibrated due to
the volume moves, and we simply read the pressure from the
vapor box.

\subsection{Simulations at Lower Temperatures}
When we have a decent result for 475 K, we start thinking
about calculating another state point at a lower
temperature. To map out the coexistence curve we might go
down in 50~K increments to 275~K. We could go to lower
temperatures but these simulations will take longer to
equilibrate.

To keep the files separate, make a directory for each
temperature. Copy in the {\bf file\_input} and {\bf
  file\_restart} files. Remember to change the temperature
in the copied {\bf file\_input} file. We will again have to
equilibrate our simulation before we begin collecting data.
This is done mainly via volume and swap moves, as the vapor
(and to a lesser extent, the liquid) density generally has a
strong temperature dependence, especially near the critical
point. Also, according to the lever rule, the relative
proportions of molecules in the vapor and liquid boxes will
change (this is the same phenomena, just expressed
differently). Thus if we hold our box sizes equal, lowering
the temperature will cause molecules to move from the vapor
box to the liquid box. The number of accepted swap moves
also will decrease since it becomes more difficult to swap
at lower temperatures. We should adjust {\bf pmswap} to
bring the number back up to between 1 and 10 accepted moves per 10 cycles.

To ensure that we keep around 50 molecules in the vapor
phase, we will have to rescale the vapor box lengths.
Compare the current number of molecules in the vapor phase
with the target amount. Say that after 10000 cycles at 425 K
we have 12 molecules in the vapor phase. We have
$\frac{50}{12} \approx 4$ times too few molecules, so we
want to increase the volume by a factor of 4. Since $V=L^3$,
we need to uniformly increase each box length ($L$) by
$4^{1/3}$ or 1.587. After the desired box length is
determined, we edit the restart file, {\bf file\_restart},
to reflect this. Note that changing the values in {\bf
  file\_input} will have no effect on the simulation if {\bf
  linit} is set to FALSE! Now we equilibrate again and check
to ensure that the vapor phase contains close to 50
molecules. From here, we start a production run to collect
data at 425 K. We repeat this process for 375 K, 325 K, {\it
  etc}. At some point, the amount of accepted swaps will
become so low that it will not be possible to have one
accepted per every 10 cycles. At this point we can increase
the number of trial positions investigated by increasing
{\bf nchoi1} and {\bf nchoi}; however, this is at an
increased computational cost. At 275 K, we might have to
live with, for example, only one accepted per 20 cycles.

%%%%%%%%%%%%%%%%%%%%%%%%%%%%%%%%%%%%%%%%%%%%%%%%%%%%%%%
\section{Special Topics}
\subsection{Thermodynamic Integration in Stages}
See Maginn's paper for the details of this method. In
Topmon, here are the changes. There is a new logical
variable called lmipsw in control.inc. If TRUE, it does the
thermodynamic integration (either use NVT or use NPT with
zero percant volume move - hence NVT). Will fail for any
other ensemble. And, only checked for one component systems.

No change in {\bf file\_input} - the input file is
unaffected by the thermodynamic integration part. However,
using lmipsw requires one new file called {\bf fort.35}. The
file is as follows:
\begin{description}
\item[{\bf lwell}] set to TRUE if stages b and c are done,
  FALSE in case of stage a. For a definition of stages, see
  Maginn's paper.
\item[{\bf awell(i,j)}] input the strength of the external
  well in K in a matrix notation (interaction of i molecule
  site with j lattice site).
\item[{\bf bwell}] parameter for gaussian well width - 0.5
  might be a good number.
\item[{\bf lstagea}, {\bf lstageb}, or {\bf lstagec}] make
  one of them TRUE.
\item[{\bf etais}, {\bf lamdbais}] the final weak potential,
  and value of {\bf lambda} for that stage (see the
  paper).
\item[box length] specify the boxlength at the beginning of
  stagea and the end of stagec (full stages) in hmatrix
  notation if lsolid is TRUE and lrect is FALSE. If lsolid
  is FALSE, use the box length of the two cubic boxes.
  Cannot mix cubic/noncubic boxes - best is to use hmatrix
  notation.
\item[{\bf iratipsw}] the number of cycles in which the
  integrand is updated. If lstageb is TRUE, make sure that
  this is an integral multiple of iratp.
\item[reduced coordinates] of all the lattice sites.
\end{description}

\subsection{New Expanded Ensemble}
It does work for changing the identity of one molecule
(e.g., from ideal to a full molecule). e.g., for growing a
solute molecule in solvent in a one-box NPT or NVT ensemble.
It may work for Gibbs (swapping that molecule in one of its
identities to the other box and then growing it) - but not
extensively tested recently. This part of the code requires
one change to the input in {\bf file\_input} - a variable
called pmexpc1 - the probabiliity of doing the new expanded
ensemble move. Other than that, there is a variable called
lexpee in control.inc. If this logical variable is TRUE,
then expanded ensmble is done and another input file called
fort.45 is required. Also there is a smax variable - maximum
number of stages that intialize the arrays.

The input file is as follows:
\begin{description}
\item[{\bf imolty}] on which ee is performed - molecule
  type. For example imolty = 1 is for the solvent, imolty =
  2 is for the solute particles already in the solution, and
  imolty = 3 is the solute particle that we will try to grow
  in the solution for this simulation.
\item[{\bf nmolty1}] number of actual types of molecules in
  the simulation. In the above example - it is 3 (1 solvent,
  1 solute molecules in full form, and 1 solute molecule
  that will go through stages).
\item[final state: {\bf fmstate}] the final state of the
  growing molecule (when it is fully grown). This determines
  the number of molecule types that are needed in {\bf
    file\_input}. In the above example, it is fmstate+2.
  Each fmstate stages have one molecule type each, and one
  for the solvent, and one for the already present solutes.
\item[weights] associated with each stage - fmstate values.
  Could start with all zero and then optimize it so that all
  stages are visited with reasonable frequency.
\item[{\bf sstate1} and {\bf sstate2}] In case expanded
  Gibbs ensemble is performed, the interdiate stages when
  the growing molecule is transferred from one box (sstate1)
  to the other (sstate2). Should be consequitive numbers. In
  case NPT or NVT expanded ensemble - make these numbers
  greater than fmstate (need to adjust smax in control such
  that these numbers are less than smax).
\item[{\bf eeratio}] In case of expanded Gibbs ensemble, the
  probability of exchanging the tagged particle (that is
  undergoing ee) at its end stages (full molecules in either
  box) with an untagged solute particle. For example, if
  stage 1 is in box 1 and stage 10 in box 2 - and at both
  these stages the tagged molecule is full molecule (just in
  different boxes) - then there should be a way to make
  another solute molecule tagged (otherwise only one
  molecule will switch between boxes). In case of NPT or NVT
  EE, make eeratio to be less than 0.
\item[{\bf mstate}] the current stage of the tagged
  molecule.
\end{description}

\subsection{Additions for Dipole Moment and Electric Field}
\noindent \textbf{dipole.dat} For an NVT simulation with
\textbf{ldielect} equal to TRUE, dipole.dat contains the
$x$, $y$ and $z$ components of the dipole moment ($\mu_x$,
$\mu_y$, and $\mu_z$) in units of $e$~\AA. These quantities
may be used to calculate the dielectric constant using the
following equation:
\begin{equation}
\epsilon = 1 + \frac{1}{4 \pi \epsilon_0} \frac{4 \pi}{3 V T} \left ( \langle \mu^2 \rangle -
 \langle \mu_x \rangle^2  - \langle \mu_y \rangle^2 - \langle \mu_z \rangle^2 \right )
\end{equation}
see Allen and Tildesley, page 161.

\noindent{\bf lelect\_field} must be set to TRUE in order for the
electric field calculation to be turned on.

\noindent In {\bf file\_input}, {\bf Elect\_field:} the
electric field strength, in units of V/\AA. The electric
field is applied in the $z$-direction. (this appears in
{\bf file\_input} after fqtemp)

In order to calculate the interaction with an electric
field, a new function (exfield) is called in sumup, energy,
and boltz. exfield.f calculates the interaction with an
electric field, given by:
\begin{eqnarray}
U_{\textrm{field}} & = & -\mu \cdot E \\
& = & -q \cdot r_z \cdot E
\end{eqnarray}

\subsection{Using Tabulated Potentials for Zeolites and MOFs}
\label{zeolite}
This sub-section explains how to use tabulated LJ 12-6 plus
Coulomb potentials for implicit solid framework structures,
where the framework is assumed to be rigid and therefore
their interactions with other species in the same simulation
box depend only on the location of the other species and can
be pre-tabulated. Note that if you want to simulate flexible
structures, then you have to prepare a suitable
\textbf{file\_restart} with all the framework atoms and this
sub-section is not relevant.

You need to set \textbf{lexzeo} to .true. and prepare a
suitable \textbf{zeolite\_in} namelist (see
\ref{zeolite_in}) to use this feature. In addition, two
extra files are needed: zeolite coordinates file,
\textbf{file\_zeocoord} and zeolite tabulated potential,
\textbf{file\_ztb}. \textbf{file\_zeocoord} can be either
CSSR, PDB, or CIF format, and should contain the coordinates
of the framework atoms along with information on the cell
parameters and interaction bead types. The following
examples assume a simulation cell constructed using $u\times
v\times w$ unit cells (you can always view it as composed of
one unit cell, but correctly specifying this information can
reduce the size of the tabulated potential). The
\textbf{rexcl} variables below are used to prevent overflow
problems, which can happen when a grid point happens to be
very close to a framework atom. The appropriate settings of
\textbf{rexcl} depend on \textbf{dgr} and
\textbf{num\_points\_interpolation} (see \ref{zeolite_in}):
we need to make sure that the radial distribution function
is zero at \textbf{rexcl} + \textbf{dgr} $\times$
\textbf{num\_points\_interpolation} / 2 so that no allowable
distances will be artifically excluded. For example, if
\textbf{dgr} and \textbf{num\_points\_interpolation} take
their default values and the RDF is zero for $r \leq
1.2$~\AA, \textbf{rexcl} can then be set to 0.6.

The SERC Daresbury Laboratory's Cambridge Structure Search
and Retrieval (CSSR) format, adapted in earlier versions of
topmon, which was possibly merged from bigmac
(\url{http://molsim.chem.uva.nl/bigmac}), were modified to
include the following lines at the very beginning (the first
character is supposed to be comment mark and will be
ignored). Note that bonding connectivities and charges are
ignored.
\begin{table}[h!]
\begin{tabular}{|l|l|l|}
\hline
LINE   & FORMAT                  & DEFINITION\\ \hline
1      & \#4 integers            & $u$, $v$, $w$, $P$ atom names.\\ \hline
2..1+P & \#string, integer, real & Atom name, bead type (as defined in \ref{atoms}), rexcl.\\ \hline
\end{tabular}
\end{table}

The Protein Data Bank (PDB) format can be used if it
contains a \textbf{CRYST1} record for the cell parameters
and the following lines are added to the very beginning of
the file:
\begin{table}[h!]
\begin{tabular}{|l|l|l|}
\hline
LINE   & FORMAT                  & DEFINITION\\ \hline
1      & \#5 intergers           & $N$ atoms, $u$, $v$, $w$, $P$ atom names.\\ \hline
2..1+P & \#string, integer, real & Atom name, bead type (as defined in \ref{atoms}), rexcl.\\ \hline
\end{tabular}
\end{table}

The Crystallographic Information File (CIF) format is
supported partially (\textbf{\_cell\_length\_\{a,b,c\},
  \_cell\_angle\_\{alpha,beta,gamma\},
  \_space\_group\_symop\_operation\_xyz and
  \_symmetry\_equiv\_pos\_as\_xyz, \_atom\_site\_label,
  \_atom\_site\_type\_symbol,
  \_atom\_site\_fract\_\{x,y,z\}}).
\textbf{\_cell\_length\_\{a,b,c\},
  \_cell\_angle\_\{alpha,beta,gamma\}} are used to provide
the \textit{unit} cell parameters (whereas
\textit{simulation} cell parameters are given in CSSR and
PDB files). Since the program does not store space groups
symmetry information,
\textbf{\_space\_group\_symop\_operation\_xyz} or
\textbf{\_symmetry\_equiv\_pos\_as\_xyz} must be included to
supply this information. The same lines, excluding the
number of framework atoms, $N$, which is now determined
automatically, should be added to the very beginning of the
file just as in the case of CSSR format.

The \textbf{file\_ztb} file will be automatically created if
it does not yet exist. It may be more convenient to first
generate it with a separate run, place the file in a shared
place, and create symbolic links to it in each simulation
directory. The tabulation can be expediated through OpenMP,
so you may enable it in CMake to compile an excutable just
for this purpose. The pre-tabulation process calculates
three values ($r^{-12}$, $r^{-6}$, and electrostatic terms)
for all grid points that are outside \textbf{rexcl} radius
of any framework atoms. The interpolation multiplies these
values with relevant $\epsilon$, $\sigma$, $q$ parameters to
give the energy. Therefore, only if you change lewald,
ltailcZeo, lshift, or rcut (but not if you change force
field parameters), do you need to re-generate the table.
\section{DPD specific notes} \label{sec:dpdunitconversions}
\subsection{DPD Unit Conversions} 
Unit conversions between simulation units and coarse grained 
units is rather straight forwards. For example, say you want 
to set up a system in which \\ \indent $r_c = 7 \AA$ 
\\ \indent {\tt Number of Beads = 500} 
\\ \indent {\tt coarse grained density = 2} 
\\ The course grained density is given by 
\begin{equation}
\rho_{C.G.} = \frac{n_{beads}}{V_{simulation}}r_c^{3}
\end{equation}
So that our simulation box volume is given by 
\begin{equation}
V_{simulation}  = \frac{n_{beads}}{\rho_{C.G.}} r_c^3
\end{equation}
finally you find the simulation box length by taking the cube root of that volume. \\ for our example system this means
\begin{equation}
L = \left(\frac{500}{2}\left(7\right)^3\right)^{\left(\frac{1}{3}\right)}=44.09\AA
\end{equation}
\\ \indent Next to convert between coarse grained pressure and the simulation pressure, see the output pressure in the output file. Find the line that reads \\ \indent {\tt pressure \t [simulation units] = } \\ Multimpy that number by $r_c^3$ to get the coarse grained pressure units. 
\subsection{Multiple bead types}
For a system with multiple bead types one must specify an atom type for each bead type in the section atoms. Furthermore one must specify an interaction parameter between each bead type in the section NONBOND
\end{document}
%%%%%%%%%%%%%%%%%%%%%%%%%%%%%%%%%%%%%%%%%%%%%%%%%%%%%%%%%%%%%%%%%%%%%%%%%%%
%%%%%%%                       END OF THE DOCUMENT
%%%%%%%%%%%%%%%%%%%%%%%%%%%%%%%%%%%%%%%%%%%%%%%%%%%%%%%%%%%%%%%%%%%%%%%%%%%
